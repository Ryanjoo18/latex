\part{Combinatorics}
basics of counting; the inclusion-exclusion principle; the pigeonhole principle; permutations and combinations; the binomial theorem; recurrence relations and linear recurrence relations; 

\chapter{Combinatorics}
\begin{itemize}
\item \href{https://rainymathboy.files.wordpress.com/2011/01/102-combinatorial-problems.pdf}{102 Combinatorial Problems From The Training of The USA IMO Team}
\end{itemize}

" Recursion and recurrence relations
" Definition of sets and functions
O Elementary probability
O Expected value and linearity of expectation
O Basic properties and definitions from graph theory, e.g. connectedness and degree of a vertex
O Definition and existence of the convex hull of a finite set of points % https://ti.inf.ethz.ch/ew/courses/CG13/lecture/Chapter%203.pdf
!! Nontrivial results from graph theory, such as Hall's marriage lemma or Turan’s theorem
\pagebreak

\section{Permutations and Combinations}
A \textbf{permutation} is an arrangement of objects in a specific order. Number of ways to permute $k$ of $n$ items:
\[ \per{n}{k} = \frac{n!}{(n-k)!} \]

A \textbf{combination} is a selection of objects without regard to the order. Number of ways to choose $k$ ok $n$ items:
\[ \com{n}{k} = \binom{n}{k} = \frac{n!}{k!(n-k)!} \]

Number of subsets of a set with $n$ elements is $2^n$.

Number of ways to choose $k$ objects from $n$ objects if repetition is allowed = $\binom{n+k-1}{k}$.

Number of paths from $(0,0)$ to $(m,n)$ going 1 unit rightwards or upwards = $\binom{m+n}{n}$.

Number of $k$-tuples of positive integers which sum equals $n$ is $\binom{n-1}{k-1}$.

Number of $k$-tuples of non-negative integers which sum equals $n$ is $\binom{n+k-1}{k-1}$.

\subsection{Stars and Bars}
The setup is the following: suppose there are three children $c_1$, $c_2$, $c_3$, and we distribute 10 identical candies among these three children. Each child can receive any number of candies, including 0. For example, one possible distribution is $(4,3,3)$: in this case, $c_1$ receives 4 candies, $c_2$ receives 3, and $c_3$ receives 3. How many ways can we distribute the candies?

The key observation is the following: we can distribute the candies by arranging them in a line, and then placing two ``bars" somewhere along the line. For example, the $(4,3,3)$ described above can be modeled by the following:
\[ \ast\ast\ast\ast | \ast\ast\ast | \ast\ast\ast \]
Each $\ast$ represents a candy, and the two location of the two bars determines the distribution of the candies. Notice that the following distribution is also possible:
\[ ||\ast\ast\ast\ast\ast\ast\ast\ast\ast\ast \]
The above diagram corresponds to the distribution $(0,0,10)$. In general, $c_1$ receives the candies left of the first bar, $c_2$ receives the candies between the two bars, and $c_3$ receives the candies right of the second bar.

So we can see that distributing candies is identical to choosing the location of the two bars to place in $10+2=12$ empty slots, hence $\binom{12}{2}$ ways.

In general, if there are $n$ candies and $k$ children, then there are $n+k-1$ slots, and we must place $k-1$ bars. The remaining $n$ candies, interspersed among the bars, represent a istribution.
Thus, the number of distributions is
\[\binom{n+k-1}{k-1}.\]
\pagebreak

\section{Combinatorial Identities}
\subsection{Pascal's Triangle}
We can observe that by means of expansion,
\begin{equation} \binom{n}{k} = \binom{n}{n-k} \end{equation}

Each number in the Pascal's triangle is a binomial coefficient. Pascal's and hockey-stick identities:
\begin{equation} 
\binom{n}{k} + \binom{n}{k+1} = \binom{n+1}{k+1} 
\end{equation}
\begin{equation}
\sum_{r=k}^{n} \binom{r}{k} = \binom{n+1}{k+1}
\end{equation}
\begin{equation}
\sum_{r=0}^{n} \binom{k+r}{r} = \binom{n+k+1}{n}
\end{equation}

We also have 
\begin{equation}
\binom{n}{k} \binom{k}{m} = \binom{n}{m} \binom{n-m}{k-m}
\end{equation}
which can be easily proven via expansion.

\begin{thrm}{Vandermonde's Identity}{}
\begin{equation}
\sum_{r=0}^{k} \binom{m}{r} \binom{n}{k-r} = \binom{m+n}{k}
\end{equation}
\end{thrm}

\subsection{Binomial Theorem}
\begin{thrm}{Binomial Theorem}{} 
For $n \in \ZZ^{+}$ and $a,b\in\RR$, 
\begin{equation}
\begin{split}
(a+b)^n &= \sum_{k=0}^n\binom{n}{k}a^{n-k}b^k\\
&= \binom{n}{0}a^n + \binom{n}{1}a^{n-1}b + \cdots + \binom{n}{n}b^n
\end{split}
\end{equation}
\end{thrm}

\begin{proof}
This can be proven using mathematical induction.
\end{proof}

\begin{corollary}
For all $n\in\ZZ^{+}$, the following equality holds:
\begin{equation}
2^n = \sum_{k=0}^{n} \binom{n}{k} = \binom{n}{0}+\binom{n}{1}+\cdots+\binom{n}{n}
\end{equation}
\end{corollary}
\begin{remark}
The above identity simply follows by $a = b = 1$.
\end{remark}
We give an alternate proof below that relates this identity to the set of subsets of a set.
\begin{proof}
Let $A$ be a set with $n$ elements, and let $A_k$ denote the subset of the power set $2_A$ containing the subsets of $A$ of size $k$. Then the sets $A_0,A_1,\dots,A_n$ partition $2^A$, which means the following
equalities hold:
\begin{align*}
2^n=|2^A| &= |A_0|+|A_1|+\cdots+|A_n| \\
&= \binom{n}{0}+\binom{n}{1}+\cdots+\binom{n}{n} = \sum_{k=0}^{n} \binom{n}{k}
\end{align*}
In other words, every subset of $A$ has a size in $\{0,1,\dots,n\}$, so to count the number of subsets of $A$, we can count the number of subsets of each size over all possible sizes.
\end{proof}

Sums:
\begin{equation}
\sum_{k=0}^{n} k \binom{n}{k} = n 2^{n-1}
\end{equation}
\begin{equation}
\sum_{k=0}^{n} k^2 \binom{n}{k} = n(n+1) 2^{n-2}
\end{equation}
\pagebreak

\section{Cardinality Rules and Principles}
In this section, we will see the formalization of counting strategies that we often take for granted: the product rule, the sum rule, and the pigeonhole principle.

\subsection{Product Rule}
% https://courses.cs.duke.edu/spring19/compsci230/Notes/lecture22.pdf

\subsubsection{Examples of counting}
Counting number of rectangles:

For a $m \times n$ grid, to form a rectangle, choose $2$ points from the $m+1$ points along the column, and choose $2$ points from the $n+1$ points along the row. Hence the number of rectangles we can form is \[ {m+1 \choose 2}{n+1 \choose 2} \]

Circle division: chords divide a circle
number of chords = n choose 2 where there are n points
number of intersection points = n choose 4 (any 4 points forms 2 chords, thus gives a unique intersection point)

\subsection{Principle of Inclusion-Exclusion}
The \textbf{principle of inclusion and exclusion} is a counting technique that computes the number of elements that satisfy at least one of several properties while guaranteeing that elements satisfying more than one property are not counted twice.

The idea behind this principle is that summing the number of elements that satisfy at least one of two categories and subtracting the overlap prevents double counting.

For two sets, 
\[ |A \cup B| = |A| + |B| - |A \cap B| \] 
where $|S|$ denotes the cardinality (i.e. number of elements) of set $S$.

For three sets,  \[ |A \cup B \cup C| = |A| + |B| + |C| - |A \cap B| - |B \cap C| - |C \cap A| + |A \cap B \cap C| \]

More generally, if $A_i$ are finite sets, then
\begin{equation} 
\begin{aligned} |\bigcup_{i=1}^{n} A_i| = &\sum_{i=1}^{n} |A_i| - \sum_{1 \le i \le j \le n} |A_i \cap A_j| + \sum_{1 \le i \le j \le k \le n} |A_i \cap A_j \cap A_k|\\
&- \cdots + (-1)^{n-1} |A_1 \cap \cdots \cap A_n|. 
\end{aligned} 
\end{equation}

\begin{exmp}{}{}
Find the number of integers from the set $\{1,2,\dots,1000\}$ which are divisible by 3 or 5.
\end{exmp}

\begin{proof}[Solution]
Let
\begin{align*}
S &= \{1,2,\dots,1000\} \\
A &= \{x\in S\:|\:x\text{ is divisible by 3}\} \\
B &= \{x\in S\:|\:x\text{ is divisible by 5}\}
\end{align*}
It follows that \[ A\cap B = \{x\in S\:|\:x\text{ is divisible by 15}\} \]
Observe that
\begin{quote}
for any two natural numbers $n$ and $k$ with $n\ge k$, the number of integers in the set $\{1,2,\dots,n\}$ which are divisible by $k$ is $\floor{\dfrac{n}{k}}$
\end{quote}
Hence we have 
\begin{align*}
|A\cup B| &= |A| + |B| - |A\cap B| \\
&= \floor{\frac{1000}{3}} + \floor{\frac{1000}{5}} - \floor{\frac{1000}{15}} \\
&= 333 + 200 - 66 = \boxed{467}
\end{align*}
\end{proof}

\subsection{Pigeonhole Principle}
\begin{thrm}{Pigeonhole Principle}{}
If $k+1$ objects are placed into $k$ boxes, then at least one box contains two or more objects. 
\end{thrm}

\begin{proof}
We use a proof by contraposition.

Suppose none of the $k$ boxes has more than one object. Then the total number of objects would be at most $k$. This contradicts the statement that we have $k + 1$ objects.
\end{proof}

\begin{thrm}{Generalised Pigeonhole Principle}{} 
If $n$ objects are placed into $k$ boxes, then there is at least one box containing at least $\ceiling{\frac{n}{k}}$ objects. 
\end{thrm}

\begin{proof}
We use a proof by contradiction.

Suppose that none of the boxes contains more than $\ceiling{\dfrac{n}{k}} - 1$ objects.

Then the total number of objects is \[k\brac{\ceiling{\frac{n}{k}}-1}\] but \[ k\brac{\ceiling{\frac{n}{k}} - 1} < k \sqbrac{\brac{\frac{n}{k} + 1} - 1} = n \]
where the inequality $\ceiling{\dfrac{n}{k}} < \dfrac{n}{k}+1$ was used.

This is a contradiction, because there are a total of $n$ objects.
\end{proof}
\pagebreak

\section{Catalan Numbers}
%https://brilliant.org/wiki/catalan-numbers/
\begin{thrm}{Catalan numbers}{}
The Catalan numbers are given by the formula
\begin{equation}
C_n = \frac{1}{n+1} \binom{2n}{n}
\end{equation}
\end{thrm}

\subsection{Dyck Paths and Acceptable Sequences}
The number of valid parenthesis expressions that consist of n right parentheses and n left parentheses is equal to the $n$-th Catalan number. 

For example, $C_3 = 5$ and there are 5 ways to create valid expressions with 3 sets of parenthesis:
\begin{itemize}
    \item ( ) ( ) ( )
    \item ( ( ) ) ( )
    \item ( ) ( ( ) )
    \item ( ( ( ) ) )
    \item ( ( ) ( ) )
\end{itemize}

Considering right parenthesis to be $+1$s, and left $-1$s, we can write this more formally as follows:

The number of sequences $a_1, \dots, a_n$ of $2n$ terms that can be formed using $n$ copies of $+1$s and $n$ copies of $-1$s whose partial sums satisfy

\subsection{Recurrence Relation; Generating Function}

\pagebreak

\section{Derangements}
A derangement is a permutation with no fixed points. That is, a derangement of a set leaves no element in its original place. For example, the derangements of $\{1,2,3\}$ are $\{2, 3, 1\}$ and $\{3, 1, 2\}$, but $\{3,2, 1\}$ is not a derangement of $\{1,2,3\}$ because $2$ is a fixed point.\\
The number of derangements of an $n$-element set is denoted $D_n$. This number satisfies the recurrences \[D_n = n \cdot D_{n - 1} + (-1)^n\]
and \[D_n = (n - 1)\cdot (D_{n - 1} + D_{n - 2})\]
and is given by the formula \[D_n = n! \sum_{k=0}^{n} \frac{(-1)^k}{k!}.\]
\pagebreak

\section{Probability}
\pagebreak

\section*{Problems}
\begin{prbm}[Langford's Problem $L(n)$]
Given the multiset\footnote{A multiset is like a set except that there may be more than one occurrence of an element.} of positive integers:
\[ \{1,1,2,2,3,3,\dots,n,n\},\]
can they be arranged in a sequence such that for $1 \le i \le n$ there are $i$ numbers between the two occurrences of $i$?
\end{prbm}

\begin{proof}

% https://link.springer.com/content/pdf/10.1007/978-3-031-13566-8_9.pdf
% https://www.youtube.com/watch?v=Lju6aYms2EA

\end{proof}
\pagebreak

\begin{prbm} 
Use a combinatorial proof to show that 
\[ \sum_{k=0}^{n} \binom{n}{k}\binom{n}{n-k} = \binom{2n}{n}. \] 
\end{prbm}

\begin{proof}
For combinatorial proofs, we begin with a story. Consider a group of $2n$ animals, where $n$ are dogs and $n$ are cats.

\textbf{RHS:} Number of ways to pick $n$ animals from a group of $2n$ animals.

For LHS, we try to understand what's going on in the summation:
\[ \sum_{k=0}^{n} \binom{n}{k}\binom{n}{n-k} = \binom{n}{0}\binom{n}{n} + \binom{n}{1}\binom{n}{n-1} + \cdots \]

We see that each term looks like a case. For example, for the first term, pick $0$ items from the first group, and pick $n$ items from the second group. This shows that if we want to pick $n$ animals, we can pick $k$ dogs and $n-k$ cats.

\textbf{LHS:} Consider all cases where we pick $k$ dogs and $n-k$ cats.

$\therefore$ LHS is the same as RHS as they both count the same number of things. Hence proven.
\end{proof}
\pagebreak

\begin{prbm} 
Evaluate 
\[ S = {n \choose 1} + 2 {n \choose 2} + 3 {n \choose 3} + \cdots + n {n \choose n}. \] 
\end{prbm}

\begin{proof}[Solution]
Writing the sum backwards yields 
\begin{align*} 
S &= n {n \choose n} + (n-1) {n \choose n-1} + \cdots + {n \choose 1} \\&= n {n \choose 0} + (n-1) {n \choose 1} + \cdots + {n \choose n-1} 
\end{align*} 
Add this to the original series gives us 
\begin{align*}
2S &= n \left[{n \choose 0} + {n \choose 1} + \cdots + {n \choose n}\right]\\ 
2S &= n 2^n \\
\Aboxed{S &= n 2^{n-1}}
\end{align*}
\end{proof}
This is the proof of the sum
\[ \sum_{k=0}^{n} k \binom{n}{k} = n 2^{n-1} \]
\pagebreak

\begin{prbm}[USAMO 2005]
Legs $L_1, L_2, L_3, L_4$ of a square table each have length $n$, where $n$ is a positive integer. For how many ordered 4-tuples $(k_1, k_2, k_3, k_4)$ of non-negative integers can we cut a piece of length $k_i$ from the end of leg $L_i \; (i=1,2,3,4)$ and still have a stable table?

(The table is stable if it can be placed so that all four of the leg ends touch the floor. Note that a cut leg of length $0$ is permitted.)
\end{prbm}

\begin{proof}[Solution]
The table is stable if $k_1+k_3=k_2+k_4$. Let this common value be $k$ such that that $k_1+k_3=k_2+k_4=k$. Let $c_k$ be the number of ways to make the table stable for each value of $k$. We want to find $\sum_{k=0}^{2n}c_k$.

Note that each table leg is at least 0 and at most $n$, hence we'll break this into two sums so that it's easier to handle:
\[\sum_{k=0}^n c_k+\sum_{k=n+1}^{2n}c_k\]

\textbf{Case 1}: If $0\le k\le n$, there are $k+1$ ways to partition $k_1$ and $k_3$, and another $k+1$ ways to partition $k_2$ and $k_4$. There are $(k+1)^2$ ways to partition $k_i$ in this interval. Hence 
\[\sum_{k=0}^n (k+1)^2\]

\textbf{Case 2}: If $n+1\le k\le 2n$, each of the $k_i$ is at most $n$ and at least $0$. There are $(2n-k+1)^2$ ways to partition the $k_i$ in this interval. Hence
\[\sum_{k=n+1}^{2n}(2n-k+1)^2\]

Evaluating the sum gives us $\boxed{\frac{(n+1)(2n^2+4n+3)}{3}}$.
\end{proof}
