\part{Complex Analysis}
\chapter{Functions}
\textbf{Readings:} Visual Complex Analysis by Needman

\section{Complex Functions}
\subsection{Exponential function}
We have Euler's formula: $e^{i\theta} = \cos\theta + i\sin\theta$. We can extend this to the complex exponential function $e^z$.

\begin{defn}{Complex exponential function}{}
For $z=x+iy$ the \vocab{complex exponential function} is defined as
\[ e^z = e^{x+iy} = e^xe^{iy} = e^x(\cos y + i\sin y). \]
\end{defn}

In this definition $e^x$ is the usual exponential function for a real variable $x$. It is easy to see that all the usual rules of exponents hold:
\begin{itemize}
\item $e^0 = 1$

\item $e^{z_1+z_2} = e^{z_1} e^{z_2}$

\item $(e^z)^n = e^{nz}$ for positive integers $n$.

\item $(e^z)^{-1} = e^{-z}$

\item $e^z \neq 0$

It will turn out that the property $\dv{e^z}{z}=e^z$ also holds, but we can't prove this yet because we haven't defined what we mean by the complex derivative d/dz. 

\item $|e^{i\theta}| = 1$
\begin{proof}
\[ |e^{i\theta}| = |\cos\theta+i\sin\theta| = \sqrt{\cos^2\theta+\sin^2\theta} = 1 \]
\end{proof}

\item $|e^{x+iy}| = e^x$
(as usual z = x + iy and x, y are real).

\item The path $e^{it}$ for $0<t<\infty$ wraps counterclockwise around the unit circle. It does so infinitely many times.
\end{itemize}

\subsection{Complex functions as mappings}
\subsection{Function $\arg z$}
3.3.1 Many-to-one functions . . . . . . . . . . . . . . . . . . . . . . . . . . 25
3.3.2 Branches of arg(z) . . . . . . . . . . . . . . . . . . . . . . . . . . . . 26
3.3.3 The principal branch of arg(z) . . . . . . . . . . . . . . . . . . . . . 28
\subsection{Branches and branch cuts}
\subsection{Function $\log z$}
3.5.1 Figures showing w = log(z) as a mapping . . . . . . . . . . . . . . . 30
3.5.2 Complex powers . . . . . . . . . . . . . . . . . . . . . . . . . . . . . 32

\section{Analytic Functions}
\subsection{The derivative: preliminaries}
\subsection{Open disks, open deleted disks, open regions}
\subsection{Limits and continuous functions}
4.3.1 Properties of limits . . . . . . . . . . . . . . . . . . . . . . . . . . . . 36
4.3.2 Continuous functions . . . . . . . . . . . . . . . . . . . . . . . . . . . 36
4.3.3 Properties of continuous functions . . . . . . . . . . . . . . . . . . . 37
\subsection{The point at infinity}
4.4.1 Limits involving infinity . . . . . . . . . . . . . . . . . . . . . . . . . 38
4.4.2 Stereographic projection from the Riemann sphere . . . . . . . . . . 39
\subsection{Derivatives}
4.5.1 Derivative rules . . . . . . . . . . . . . . . . . . . . . . . . . . . . . . 40
\subsection{Cauchy-Riemann equations}
4.6.1 Partial derivatives as limits . . . . . . . . . . . . . . . . . . . . . . . 41
4.6.2 The Cauchy-Riemann equations . . . . . . . . . . . . . . . . . . . . 42
4.6.3 Using the Cauchy-Riemann equations . . . . . . . . . . . . . . . . . 43
4.6.4 f0(z) as a 2 × 2 matrix . . . . . . . . . . . . . . . . . . . . . . . . . . 44
4.7 Geometric interpretation \& linear elasticity theory . . . . . . . . . . . . . . 45
4.8 Cauchy-Riemann all the way down . . . . . . . . . . . . . . . . . . . . . . . 46
4.9 Gallery of functions . . . . . . . . . . . . . . . . . . . . . . . . . . . . . . . 47
4.9.1 Gallery of functions, derivatives and properties . . . . . . . . . . . . 47
4.9.2 A few proofs . . . . . . . . . . . . . . . . . . . . . . . . . . . . . . . 51
4.10 Branch cuts and function composition . . . . . . . . . . . . . . . . . . . . . 52
4.11 Appendix: Limits . . . . . . . . . . . . . . . . . . . . . . . . . . . . . . . . . 54
4.11.1 Limits of sequences . . . . . . . . . . . . . . . . . . . . . . . . . . . . 54
4.11.2 limz→z0
f(z) . . . . . . . . . . . . . . . . . . . . . . . . . . . . . . . . . 56
4.11.3 Connection between limits of sequences and limits of functions . . . 56
