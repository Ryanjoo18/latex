\part{Abstract Algebra}
\chapter{Group Theory}
\textbf{Readings:}
\begin{itemize}
\item \href{https://www.jmilne.org/math/CourseNotes/GT.pdf}{Group Theory by J.S. Milne}
\item \href{https://people.maths.ox.ac.uk/flynn/genus2/sheets0405/grfnotes1011.pdf}{Introduction to Groups, Rings and Fields by Oxford}
\item \href{https://people.maths.bris.ac.uk/~majm/bib/talks/grouptheory.pdf}{Math 33300: Group Theory}
\item \href{https://web.evanchen.cc/notes/SJSU179.pdf}{Math 179: Graph Theory}
\item \href{https://www.cs.cmu.edu/~15251/notes/polynomials-ecc.pdf}{Groups, Fields and Polynomials}
\end{itemize}
% https://people.seas.harvard.edu/~madhusudan/MIT/FT05/scribe/lect02.pdf
% https://people.seas.harvard.edu/~madhusudan/MIT/ST15/scribe/lect04.pdf

\section{Binary Operations}
\begin{defn}{Binary operation}{}
A binary operation $\ast$ on a set $S$ is a map $\ast: S \times S \to S$. We write $a \ast b$ for the image of $(a, b)$ under $\ast$.
\end{defn}

So a binary operation takes two inputs $a$ and $b$ from $S$ in a given order and returns a single output $a \ast b$ which importantly has to be in $S$. Standard examples include addition, multiplication and
composition but there are many other examples as well.

\begin{exmp}{}{}
The following are examples of binary operations.
\begin{itemize}
\item $+, -, \times$ on $\RR$; $\divisionsymbol$ is not a binary operation on $\RR$ as, for example $1 \divisionsymbol 0$ is undefined;
\item $\wedge$, the cross product, on $\RR^3$;
\item $\min$ and $\max$ on $\NN$;
\item $\circ$, composition, on the set Sym(S) of bijections of a set $S$ to itself.
\end{itemize}
\end{exmp}

A binary operation $\ast$ on a set $S$ is said to be \textbf{associative} if, for any $a, b, c \in S$,
\[ (a \ast b) \ast c = a \ast (b \ast c). \]

In particular, this means an expression such as $a_1 \ast a_2 \ast \cdots \ast$ an always yields the same result, irrespective of how the individual parts of the calculation are performed.

A binary operation $\ast$ on a set $S$ is said to be \textbf{commutative} if, for any $a, b \in S$,
\[ a \ast b = b \ast a. \]

An element $e \in S$ is said to be an \textbf{identity element} (or simply an identity) for an operation $\ast$ on $S$ if, for any $a \in S$,
\[ e \ast a = a = a \ast e. \]

\begin{proposition}[Uniqueness of identity]
Let $\ast$ be a binary operation on a set $S$ and let $a \in S$. If an identity $e$ exists then it is unique.
\end{proposition}
\begin{proof}
Suppose that $e_1$ and $e_2$ are two identities for $\ast$. Then
\[ e_1 \ast e_2 = e_1 \quad \text{as }e_2\text{ is an identity;} \]
\[ e_1 \ast e_2 = e_2 \quad \text{as }e_1\text{ is an identity.} \]
Hence $e_1 = e_2$.
\end{proof}

If an operation $\ast$ on a set $S$ has an identity $e$ and $a \in S$, then we say that $b \in S$ is an \textbf{inverse} of $a$ if
\[ a \ast b = e = b \ast a. \]

\begin{proposition}[Uniqueness of inverse]
Let $\ast$ be an associative binary operation on a set $S$ with an identity $e$ and let $a \in S$. Then an inverse of $a$, if it exists, is unique.
\end{proposition}
\begin{proof}
Suppose that $b_1$ and $b_2$ are inverses of $a$. Then
\[ b_1 \ast (a \ast b_2) = b_1 \ast e = b_1; \]
\[ (b_1 \ast a) \ast b_2 = e \ast b_2 = b_2. \]
By associativity $b_1 = b_2$.
\end{proof}

\begin{notation}
If $\ast$ is an associative binary operation on a set $S$ with identity $e$, then the inverse of $a$ (if it exists) is written $a^{-1}$.
\end{notation}

\begin{exmp}{}{}
\begin{itemize}
\item $+$ on $\RR$ is associative, commutative, has identity 0 and $x^{-1} \coloneqq -x$ for any $x$; $-$ on $\RR$ is not associative or commutative and has no identity; $\times$ on $\RR$ is associative, commutative, has identity 1 and $x^{-1} \coloneqq \frac{1}{x}$ for any non-zero $x$.
\item $\wedge$ on $\RR$ is not associative or commutative and has no identity;
\item $\min$ on $\NN$ is both associative and commutative but has no identity; $\max$ on $\NN$ is both associative and commutative and has identity 0 (being the least element of $\NN$) though no positive integer has an inverse;
\item $\circ$ is associative, but not commutative, with the identity map $x \to x$ being the identity element and as permutations are bijections they each have inverses.
\end{itemize}
\end{exmp}
\pagebreak

\section{Group Axioms}
A \textbf{group} is an algebraic structure that captures the idea of symmetry without an object.

\begin{defn}{Group}{}
A \textbf{group} is a pair $(G,\ast)$, where $G$ is a set and $\ast$ is a binary operation on $G$ satisfying the following \textbf{group axioms}:
\begin{enumerate}[label=G\arabic*]
\item \label{g1_assoc} \textbf{Associativity}: For all $a,b,c \in G$, $a \ast (b \ast c)=(a \ast b) \ast c$
\item \label{g2_id} \textbf{Identity}: There exists an identity element $1_G \in G$ such that for all $a\in G$, $a \ast 1_G = 1_G \ast a = a$
\item \label{g3_inver} \textbf{Invertibility}: For all $a \in G$, there exists a unique inverse $a^{-1} \in G$ such that $a \ast a^{-1} = a^{-1} \ast a = 1_G$
\end{enumerate}
The final axiom is rather trivial -- \textbf{Closure}: For all $a,b,c \in G$, $a\ast b\in G$
\end{defn}

$G$ is \vocab{abelian}\footnote{After the Norwegian mathematician Niels Abel (1802--1829)} if the operation is commutative; it is \textbf{non-abelian} if otherwise.

\begin{notation}
A group $(G,\ast)$ is usually simply denoted by $G$.
\end{notation}

\begin{notation}
We abbreviate $a \ast b$ to just $ab$. Also, since the operation $\ast$ is associative, we can omit unnecessary parentheses: $(ab)c = a(bc) = abc$.
\end{notation}

\begin{notation}
For any $g \in G$ and $n \in \NN$ we abbreviate $g^n = \underbrace{g \ast \cdots \ast g}_{n\text{ times}}$.
\end{notation}

\begin{exmp}{Additive integers}{}
The pair $(\ZZ,+)$ is a group. Note that
\begin{itemize}
\item The element $0 \in \ZZ$ is an identity: $a+0=0+a=a$ for any $a$.
\item Every element $a \in \ZZ$ has an additive inverse: $a+(-a)=(-a)+a=0$.
\end{itemize}
We call this group $\ZZ$.
\end{exmp}

\begin{exmp}{Addition mod $n$}{}
Let $n > 1$ be an integer, and consider the residues (remainders) modulo $n$. These form a group under addition. We call this the cyclic group of order $n$, and denote it as $\ZZ/n\ZZ$, with elements $0, 1, \dots, n-1$. The identity is 0.
\end{exmp}

\begin{proposition}
Cancellation laws hold in groups.
\end{proposition}
\begin{proof}
By \cref{g3_inver},
\[ ab=ac \implies b=c,\quad ba=ca\implies b=c \]
by multiplying $a^{-1}$ on LHS or RHS. 
\end{proof}

\begin{proposition}[Inverse of products]
For $a,b \in G$, $(ab)^{-1} = b^{-1}a^{-1}$.
\end{proposition}
\begin{proof}
Direct computation. We have
\[ (ab)(b^{-1}a^{-1}) = a(bb^{-1})a^{-1} = aa^{-1} = 1_G. \]
Similarly, 
\[ (b^{-1}a^{-1})(ab) = 1_G. \]
Hence equating both gives us $(ab)^{-1} = b^{-1}a^{-1}$.
\end{proof}

\begin{proposition}[Left multiplication is a bijection]
For a group $G$, pick a $g \in G$. Then the map $G \to G$ given by $x \mapsto gx$ is a bijection.
\end{proposition}
\begin{proof}
Check this by showing injectivity and surjectivity directly.
\end{proof}

\begin{defn}{Order}{}
The \vocab{order} of a finite group $G$ is the number of elements in $G$, denoted by $|G|$.

The order of $a \in G$ is the least $k$ such that $a^k = 1_G$. This is consistent with the definition of order of a group, as the order of $a$ is the order of the subgroup generated by $a$.
\end{defn}

\begin{defn}{Subgroup}{}
A \vocab{subgroup} $H$ of a group $G$ is a \emph{subset} of $G$ which is a group under the operation of G restricted to $H$. We write $H \le G$. In particular, a subset $H\subseteq G$ is a subgroup if it is closed under the operation of $G$.
\end{defn}

\begin{defn}{Coset}{}
A (left) \vocab{coset} of a subgroup $H \le G$ is a set $aH = \{ah \mid h \in H\}$.
\end{defn}

Two (left) cosets $aH$ and $bH$ are either disjoint or equal. 

Since multiplication is injective, the cosets of $H$ are the same size as $H$, and thus $H$ partitions $G$ into equal-sized parts.
\pagebreak

\section{Isomorphism}
\begin{defn}{Isomorphism}{}
Let $G = (G,\ast)$ and $H = (H,\star)$ be. A bijection $\phi: G \to H$ is called an \textbf{isomorphism} if, for all $g_1, g_2 \in G$,
\[ \phi(g1 \ast g2) = \phi(g1) \star \phi(g2). \]
If there exists an isomorphism from $G$ to $H$, $G$ and $H$ are \textbf{isomorphic}, denoted by $G \cong H$.
\end{defn}
\begin{remark}
Note that in this definition, the left-hand side $\phi(g1 \ast g2)$ uses the operation of $G$ while the right-hand side $\phi(g1) \star \phi(g2)$ uses the operation of $H$.
\end{remark}

\begin{exmp}{$\ZZ\cong10\ZZ$}{}
Consider the two groups
\[ \ZZ = (\{\dots, -2, -1, 0, 1, 2, \dots\}, +) \] and
\[ 10\ZZ = (\{\dots, -20, -10, 0, 10, 20, \dots\}, +). \]
These groups are ``different", but only superficially so --- you might even say they only differ in the names of the elements.

Formally, the map
\[ \phi: \ZZ \to 10\ZZ \text{ by } x \mapsto 10x \]
is a bijection of the underlying sets which respects the group operation. In symbols,
\[ \phi(x+y) = \phi(x) + \phi(y). \]
In other words, $\phi$ is a way of re-assigning names of the elements without changing the structure of the group.
\end{exmp}
\pagebreak

\section{Lagrange's theorem}
An important result relating the order of a group with the orders of its subgroups is Lagrange's theorem.

\begin{thrm}{Lagrange's theorem}{}
If $G$ is a finite group and $H$ is a subgroup of $G$, then $|H|$ divides $|G|$.
\end{thrm}

Groups of small order (up to order 8). Quaternions. Fermat-Euler theorem
from the group-theoretic point of view.

\begin{thrm}{Fermat's Little Theorem}{}
For every finite group $G$, for all $a \in G$, $a^{|G|} = 1_G$.
\end{thrm}
\begin{proof}
Consider the subgroup $H$ generated by $a$: $H = \{a^i \mid i \in \ZZ\}$. Since $G$ is finite, the infinite sequence $a^0 = 1_G, a^1, a^2, a^3, \dots$ must repeat, say $a^i = a^j, i < j$. Let $k=j-i$. Multiplying both sides by $a^{-i} = (a^{-1})^i$, we get $a^{j-i} = a^k = 1_G$. Suppose $k$ is the least positive integer for which this holds. Then $H = \{a_0, a_1, a_2, \dots, a^{k-1}\}$, and thus $|H| = k$. By Lagrange’s Theorem, $k$ divides $|G|$, so $a^{|G|} = (a^k)^\frac{|G|}{k} = 1_G$.
\end{proof}

\section{Group actions}
Group actions; orbits and stabilizers. Orbit-stabilizer theorem. Cayley’s theorem (every group is isomorphic to a subgroup of a permutation group). Conjugacy classes. Cauchy’s theorem.

\section{Quotient groups}
Normal subgroups, quotient groups and the isomorphism theorem

\section{Matrix groups}
The general and special linear groups; relation with the M¨obius group. The orthogonal and special orthogonal groups. Proof (in R3) that every element of the orthogonal group is the product of reflections and every rotation in R3 has an axis. Basis change as an example of conjugation.

\section{Permutations}
Permutations, cycles and transpositions. The sign of a permutation. Conjugacy in Sn and in An. Simple groups; simplicity of A5.

\chapter{Ring Theory}
\textbf{Readings:}
\begin{itemize}
\item \href{https://brilliant.org/wiki/ring-theory/}{Ring Theory by Brilliant}
\item \href{https://math.berkeley.edu/~gmelvin/math113su14/math113su14notes2.pdf}{Ring Theory (Math 113) by UC Berkeley}
\end{itemize}

\section{Definition}
A ring is just a set where you can add, subtract, and multiply. In some rings you can divide, and in others you can’t. There are many familiar examples of rings, the main ones falling into two camps: ``number systems" and ``functions".

\begin{defn}{Ring}{}
A ring is a set $R$ endowed with two binary operations, addition and multiplication, denoted $+$ and $\times$, with elements $0,1\in R$, which maps $+: R \times R \to R$ and $\times: R \times R \to R$, subject to three axioms:
\begin{enumerate}[label=R\arabic*]
\item $(R,+)$ is an abelian group with identity 0,
\item $(R,\times)$ is a commutative semigroup, i.e. $a \times (b \times c) = (a \times b) \times c, a \times 1 = 1 \times a = a$, and $a \times b = b \times a$ for all $a, b, c \in R$,
\item Distributivity: $a \times (b + c) = a \times b + a \times c$ for all $a, b, c \in R$.
\end{enumerate}
\end{defn}

Examples of rings:
\begin{itemize}
\item $\ZZ$: the integers $\dots,-2,-1,0,1,2,\dots$ with usual addition and multiplication, form a ring. Note that we cannot always divide, since 1/2 is no longer an integer.

\item $2\ZZ$: the even integers $\dots,-4,-2,0,2,4,\dots$

\item $\ZZ[x]$: this is the set of polynomials whose coefficients are integers. 

It is an extension of $\ZZ$, in the sense that we allow all the integers, plus an “extra symbol” $x$, which we are allowed to multiply and add, giving rise to $x^2$, $x^3$, etc., as well as $2x$, $3x$, etc. Adding up various combinations of these gives all the possible integer polynomials.

\item $\ZZ[x,y,z]$: polynomials in three variables with integer coefficients. 

This is an extension of the previous ring. In fact you can continue adding variables to get larger and larger rings.

\item $\ZZ/n\ZZ$: integers mod $n$. 

These are equivalence classes of the integers under the equivalence relation “congruence mod n”. If we just think about addition (and subtraction), this is exactly the cyclic group of order $n$. However, when we call it a ring, it means we are also using the operation of multiplication.

\item Q, R, C

\end{itemize}

Ideals, homomorphisms, quotient rings, isomorphism theorems. Prime and maximal ideals. Fields. The characteristic of a field. Field of fractions of an
integral domain.
Factorization in rings; units, primes and irreducibles. Unique factorization in principal ideal domains, and
in polynomial rings. Gauss’ Lemma and Eisenstein’s irreducibility criterion.
Rings $\ZZ[\alpha]$ of algebraic integers as subsets of $\CC$ and quotients of $\ZZ[x]$. Examples of Euclidean domains and
uniqueness and non-uniqueness of factorization. Factorization in the ring of Gaussian integers; representation of integers as sums of two squares.
Ideals in polynomial rings. Hilbert basis theorem

\chapter{Field Theory}
\section{Field Axioms}
\begin{defn}{Field}{}
A field is a ring $R$ that satisfies the following extra properties
\begin{itemize}
\item $0 \neq 1$
\item every non-zero element of $R$ has a multiplicative inverse (or reciprocal): if $r \in R$ and $r \neq 0$, then there exists $s \in R$ such that $rs=1$; in other words: $R \setminus\{0\}$ is a group under $\times$ with identity 1.
\end{itemize}
\end{defn}

Non-example of a field: Z. Indeed, $3 \in Z$ and $7 \in Z$, but there is no integer x such that $3x = 7$, so $3/7 \notin \ZZ$.
However, Q, R, C are fields. (Z/nZ is a field if and only if n is prime.)

\begin{exmp}{$\ZZ^+$}{}
The set of \textbf{positive integers} $\ZZ^+$ is not a field because, for example, 0 is not a positive integer, for no positive integer $n$ is $-n$ a positive integer, for no positive integer $n$ except 1 is $n^{-1}$ a positive integer.
\end{exmp}

\begin{exmp}{$\ZZ$}{}
The set of \textbf{integers} $\ZZ$ is not a field because for an integer $n$, $n^{-1}$ is not an integer unless $n=1$ or $n=-1$.
\end{exmp}

\begin{exmp}{$\QQ$}{}
The set of \textbf{rational numbers} $\QQ$ is a field.
\end{exmp}

\begin{proposition}
Suppose $K$ is a field and $X \subseteq K$ is a subset of $K$, with the following properties:
\begin{itemize}
\item $0, 1 \in X$,
\item if $x, y \in X$, then $x+y, x-y, x \times y \in X$; and if $y \neq 0$, then $\frac{x}{y} \in X$.
\end{itemize}
Then $X$ is a field.
\end{proposition}
\begin{proof}
By assumption, $X$ is closed under addition and multiplication. Moreover, $X$ is clearly a ring, because $X$ inherits all the axioms from $K$. Finally, $0 \neq 1$, and if $0 \neq x \in X$, then $x^{-1} \in X$ by assumption. Therefore, $X$ is a field.
\end{proof}
We call $X$ a \textbf{subfield} of $K$.

\chapter{Galois Theory}
\textbf{Readings:}
\begin{itemize}
\item \href{https://www.maths.ed.ac.uk/~tl/gt/gt.pdf}{Notes by Tom Leinster}
\end{itemize}

\chapter{Category Theory}
\textbf{Readings:}
\begin{itemize}
\item \href{https://arxiv.org/pdf/1612.09375.pdf}{Basic Category Theory, by Tom Leinster}
\end{itemize}


