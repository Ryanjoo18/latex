\part{Thermodynamics}
\chapter{Classical thermodynamics}
\section{The zeroth law of Thermodynamics}
Concepts of thermal equilibrium and reversible pro-
cesses; internal energy, work and heat; Kelvin’s tem-
perature scale; entropy; open, closed, isolated systems;
first and second laws of thermodynamics. Kinetic the-
ory of ideal gases: Avogadro number, Boltzmann factor
and gas constant; translational motion of molecules and
pressure; ideal gas law; translational, rotational and os-
cillatory degrees of freedom; equipartition theorem; in-
ternal energy of ideal gases; root-mean-square speed of
molecules. Isothermal, isobaric, isochoric, and adiabatic
processes; specific heat for isobaric and isochoric pro-
cesses; forward and reverse Carnot cycle on ideal gas and
its efficiency; efficiency of non-ideal heat engines.

Equation of state
\begin{equation}
pV = nRT
\end{equation}
where $n$ is the amount of substance in moles, $R = 8.31$ \unit{J.K^{-1}.mol^{-1}} is the molar gas constant. $N_A = 6.02 \times 10^{23}$ \unit{mol^{-1}} is the Avogardo number.

This equation can also be written as
\[ pV = Nk_BT \]
where $k_B = \frac{R}{N_A} = 1.38 \times 10^{-23}$ \unit{J.K^{-1}} is the Boltzmann constant.

\begin{thrm}{0th Law of Thermodynamics}{}
If A, B and C are different thermodynamical systems and A is in thermodynamical equilibrium with B and B is in thermodynamical equilibrium with C, then A is in thermodynamical equilibrium with C.
\end{thrm}

\section{The first law of Thermodynamics}
We saw from the zeroth law that there are two kinds of energy that can be transferred between a thermodynamical system and its surroundings:
\begin{enumerate}
\item work $W$ by mechanical contact
\item heat $Q$ by thermal contact
\end{enumerate}

\begin{thrm}{1st Law of Thermodynamics}{}
The internal energy of an isolated system is conserved under any thermodynamical change.

Under any thermodynamical change,
\begin{equation}
U = Q + W
\end{equation}
\end{thrm}

where $U$ is the internal energy of the system (function of state),$ $Q is heat added to the system, $W$ is the work done \emph{on} the system.

According to the first law we thus have $Q_\text{surr} = Q$ and $W_\text{surr} = W$, where the subscript ``surr" indicates the system’s surroundings.

\subsection{Internal energy}
The energy $E_l$ of a particle $l$, according to the fundamental laws of physics, is either in kinetic or potential form, which we will write $E_{l,K}$ and $E_{l,P}$ respectively. 

We introduce an internal state energy $E_{l,I}$, which is composed of intra-molecular kinetic and potential energies reflecting the structure of the molecule.

So the total internal energy of a system is simply
\[ E = \sum_l (E_{l,K} + E_{l,P} + E_{l,I}) \]

\pagebreak

\chapter{Heat transfer and phase transitions}
Phase transitions (boiling, evaporation, melting, sublimation) and latent heat; saturated vapour pressure, relative humidity; boiling; Dalton’s law; concept of heat conductivity; continuity of heat flux.

Thermal conduction:
\begin{thrm}{Fourier's Law}{}
\begin{equation}
{\frac{Q}{t} = \frac{kA\Delta T}{L}
}\end{equation}
where $\frac{Q}{t}$ is rate of thermal energy transfer, $k$ is thermal conductivity, $A$ is barrier cross-sectional area, $\Delta T$ is temperature difference, $L$ is barrier thickness.
\end{thrm}
\pagebreak

