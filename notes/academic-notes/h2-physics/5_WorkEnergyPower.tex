\section{Work, Energy and Power}
\subsection{Work}
\begin{defn}{Work done by constant force}{}
Product of the force and displacement \underline{in the direction of the force}.
\begin{equation} W = F s \cos \theta \end{equation}
\end{defn} 

\begin{itemize}
\item Work done by a \textbf{variable force} is given by
\begin{equation} W = \int F \dd{s} \end{equation}
Graphically, work done is area under force-displacement graph.

\item Work done to deform (stretch/compress) a material is stored as elastic potential energy in the material.
\begin{equation}
    U = \frac{1}{2}Fx = \frac{1}{2}kx^2
\end{equation}

\item Work done by a \textbf{gas} which is expanding against a constant external pressure: 
\begin{equation}
W = p \Delta V
\end{equation}

Graphically, work done is the area under a pressure-volume graph.
\[ W = \int p \dd{V} \]
\end{itemize}
\pagebreak

\subsection{Energy conversion and conservation}
\begin{defn}{Principle of Conservation of Energy}{}
Energy can neither be created nor destroyed, but can be transformed from one form to another, and transferred from one body to another. Total energy in a closed system is always constant.
 \begin{equation}
(E_k+E_p)_i + W = (E_k+E_p)_f
\end{equation}
\begin{remark}
Work done by dissipative forces is \emph{negative} as the forces act in opposite direction to displacement.
\end{remark}
\end{defn}

\textbf{Gravitational potential energy} is energy stored due to height raised.
\begin{equation}
\mathrm{GPE} = mgh
\end{equation}

\deriv{See Appendix for the derivation.}

\textbf{Kinetic energy} is energy possessed by an object due to motion.
\begin{equation} 
\mathrm{KE} = \frac{1}{2}mv^2 = \frac{p^2}{2m} 
\end{equation}

\deriv{See Appendix for the derivation.}

\textbf{Elastic potential energy} is energy stored in an object when it is deformed.
\begin{equation}
\mathrm{EPE} = \frac{1}{2} Fx = \frac{1}{2} k x^2 
\end{equation}

\begin{defn}{Work-Energy Theorem}{}
Net work done by a force on a body is equal to the change in kinetic energy of the body.
\begin{equation} W = \Delta \mathrm{KE} \end{equation}
\end{defn} 

The relationship between conservative force $F$ and potential energy $U$ is
\begin{equation} F = -\dv{U}{x} \iff U = -\int F \dd{x} \end{equation}
\begin{remark}
A conservative force is one where work done by the force is independent of its path.
\end{remark}
\pagebreak

\subsection{Power}
\begin{defn}{Power}{}
Rate at which work is done; rate at which energy is transferred.
\begin{equation}
P = \dv{W}{t}
\end{equation}
\end{defn}

\vocab{Instantaneous power} $P$ when a constant force $F$ acts on an object with velocity $v$ is given by
\begin{equation} P = Fv \end{equation}
\begin{remark}
This means power is the product of a force and velocity in the direction of the force.
\end{remark}

\vocab{Average power} $P_{\mathrm{avg}}$ when a constant force $F$ acts on an object with average velocity $v_{\mathrm{avg}}$ is given by
\begin{equation} P_{\mathrm{avg}} =  F v_{\mathrm{avg}} \end{equation}

\subsection{Efficiency}
\vocab{Efficiency} $\eta$ is given by
\begin{equation} \eta = \frac{\mathrm{useful\:power/energy\:output}}{\mathrm{total\:power/energy\:input}} \times 100\% \end{equation}
\pagebreak

\subsection*{Problems}
\begin{prbm}
A hydroelectric dam has a water height of 50 \unit{m} (as measured from the bottom of the dam where water is let out). What is the rate at which water is let out to produce 50MW of electrical power? You should assume an energy conversion efficiency of the dam (from mechanical to electrical) to be $30\%$, and the density of water as 997 \unit{kg.m^{-3}}.

Assume that the dam is large enough so that the water height does not substantially change during power generation.
\end{prbm}

\begin{proof}[Solution]
By conservation of energy, the kinetic energy of water that leaves the bottom of the dam is equal to gravitational potential energy at the top of the dam.

Power generated is (efficiency) × (rate of flow) × (density) × g × (50 metres). Equating this to 50 MW produces a rate of flow of 341 \unit{m^3.s^{-1}}.
\end{proof}
\pagebreak