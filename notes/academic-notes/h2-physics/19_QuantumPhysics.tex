\section{Quantum Physics}
\subsection{Photoelectric effect}
\begin{defn}{Photoelectric effect}{}
Emission of electrons from metal surface when electromagnetic radiation of sufficiently high frequency is incident on it.
\end{defn}

Experimental observations from the photoelectric effect experiment:
\begin{enumerate}
\item No electrons are emitted if the frequency of the EM radiation is below a minimum frequency (called the threshold frequency $f_0$), regardless of the intensity of the radiation.
\item Photoelectric current is proportional to the intensity of radiation, for a fixed frequency (because the rate of emission of electrons $\propto$ rate of incidence of photons)
\item Max KE of photo-electrons depends only on the frequency and the work function $\phi$ of the metal used, not the intensity. (Note: Emitted electrons have a range of kinetic energy, ranging from zero to a certain maximum value.)
\item Emission of electrons begins instantaneously (i.e. no (measurable) time lag between emission and illumination) even if the intensity is low.
\end{enumerate}

(1), (2) and (3) cannot be explained by Classical Wave Theory of Light; they provide evidence for the particulate\footnote{particle-like} nature of EM radiation.

Failure of the classical wave theory to explain the photoelectric effect
\begin{itemize}
\item According to the “Particle Theory of Light”, EM radiation consists of a stream of particles/ photons/ discrete energy packets, each of energy hf.
\item An electron is ejected when a single photon of sufficiently high frequency, transfers ALL its energy in a discrete packet to the electron.
\item According to equation, $hf - \phi = \frac{1}{2}m_ev^2$, if the energy of the photon hf < the minimum energy required for emission ($\phi$), no emission can take place, no matter how intense the light may be. {Explains observation (1)}
\item This also explains why, (even at very low intensities), as long as $hf > \phi$, emission takes place without a time delay between illumination of the metal and ejection of electrons. {Explains observation(4)}
\end{itemize}

\subsection{Energy of a photon}
Particulate nature of electromagnetic radiation:
\begin{itemize}
\item Electromagnetic radiation can be said to be particulate in nature in addition to being waved in nature.
\item A beam of light consists of small discrete quanta of electromagnetic energy known as photons.
\item Photons transfer either all or none of their energy to another particle instantaneously, contrary to the wave theory which states that the energy transfer is continuous.
\end{itemize}

\begin{defn}{Photon}{}
A discrete packet (or quantum) of energy of an electromagnetic radiation with energy $hf$.
\end{defn}

Energy of a photon is given by
\begin{equation}
E = \hbar f
\end{equation}
where \textbf{Planck constant} $\hbar = 6.63 \times 10^{-34}\unit{J}\unit{s}$.



\subsection{Wave-particle duality}
Waves can exhibit particle-like characteristics and particles can exhibit wave-like characteristics.

de Broglie wavelength of a particle: 
\begin{equation}
\lambda = \frac{h}{p}
\end{equation}

Packets of EM radiation of wavelength $\lambda$ would therefore possess a momentum $p=\frac{h}{\lambda}$. When photons are incident on a surface, they therefore exert a force on the surface, resulting in a pressure on the surface. This pressure is known as “radiation pressure”. 

Using $KE=\frac{p^2}{2m}$,  wavelength of a particle
can be related to its KE by
\begin{equation}
\lambda = \frac{h}{\sqrt{2m(KE)}}
\end{equation}

\subsection{Energy levels in atoms}


\subsection{Line spectra}


\subsection{X-ray spectra}


\subsection{Uncertainty principle}
\begin{defn}{Heisenberg position-momentum uncertainty principle}{}
It is impossible to measure the exact position and momentum of a body at the same time. It can be expressed as
\begin{equation}
\Delta p \Delta x \ge \hbar
\end{equation}
where $\Delta p$ and $\Delta x$ denote the uncertainties in the momentum and position of the particle respectively.
\end{defn}

\subsection*{Problems}

\pagebreak