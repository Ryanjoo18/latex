\section{Measurement}
\subsection{Physical quantities and SI units}
\begin{defn}{Base quantity}{}
Physical quantity that cannot be defined in terms of other quantities.
\end{defn}

\begin{defn}{Base unit}{}
Unit which is defined without referring to other units.
\end{defn}
\begin{remark}
Base units are not to be confused with \textbf{SI units}, which refer to the set of standard units that are commonly used. For instance, the SI unit for frequency is \unit{Hz}, but the base unit is \unit{s^{-1}}.
\end{remark}

\begin{table}[H]
	\centering
	\begin{tabular}{cc} 
	\hline\hline
	\textbf{Quantity} & \textbf{Unit} \\ 
	\hline
	mass & kilogram (kg) \\
	length & metre (m) \\
	time & second (s) \\
	current & ampere (A) \\
	temperature & kelvin (K) \\
	amount of substance & mole (mol) \\
	\hline\hline
	\end{tabular}
\end{table}

\begin{defn}{Derived quantity}{}
Physical quantity derived from base quantities, can be expressed in terms of product and/or quotient of base quantities.
\end{defn}

\begin{defn}{Derived unit}{}
Unit derived from base units, can be expressed in terms of products and/or quotients of base units.
\end{defn}
\pagebreak
List of \vocab{prefixes}:
\begin{table}[H]
	\centering
	\begin{tabular}{ccc} 
	\hline\hline
	\textbf{Prefix} & \textbf{Symbol} & \textbf{Factor} \\
	\hline
	tera & T & ${10}^{12}$ \\ 
	giga & G & ${10}^{9}$ \\
	mega & M & ${10}^{6}$ \\
	kilo & k & ${10}^{3}$ \\
	deci & d & ${10}^{-1}$ \\
	centi & c & ${10}^{-2}$ \\ 
	milli & m & ${10}^{-3}$ \\
	micro & $\mu$ & ${10}^{-6}$ \\ 
	nano & n & ${10}^{-9}$ \\
	pico & p & ${10}^{-12}$ \\
	\hline\hline
	\end{tabular}
\end{table}

These are some reasonable estimates of physical quantities.
\begin{table}[H]
	\centering
	\begin{tabular}{cc} 
	\hline\hline
	\textbf{Quantity} & \textbf{Estimatation} \\ 
	\hline
	Frequency of audible sound wave & 20 Hz to 20 kHz \\ 
	Wavelength of UV radiation & $1 \times 10^{-7}$ to $1 \times 10^{-8}$ nm \\
	Mass of 30cm plastic ruler & 30 \unit{g} to 50 \unit{g} \\
	Density of atmospheric air & 1 \unit{kg.m^{-3}} \\
	\hline\hline
	\end{tabular}
\end{table}

\subsection{Dimensional analysis}
A \vocab{homogeneous equation} is an equation where all quantities have the \emph{same units}. Use SI base units to check the homogeneity of physical equations.

A physically correct equation must be homogeneous; a homogeneous equation may not be physically correct. Some reasons include:
\begin{itemize}
    \item Value of dimensionless factor may be incorrect.
    \item Missing or extra terms that may have the same unit.
\end{itemize}

\subsection{Scalars and vectors}
\begin{defn}{Scalar quantity}{}
Only has magnitude but no direction.
\end{defn}
Examples: distance, speed, energy

\begin{defn}{Vector quantity}{}
Has both magnitude and direction.
\end{defn}
Examples: displacement, velocity, force

\subsection{Vectors}
Use of trigonometry, Sine Rule and Cosine Rule is relevant.
\subsubsection{Vector addition}
Vectors can be added via
\begin{enumerate}
\item \textbf{Triangle method}

\item \textbf{Parallelogram method}

\end{enumerate}

\subsubsection{Vector subtraction}
Used to determine the \emph{change} in a certain vector quantity.

\subsubsection{Resolving vector}
Represent a vector as two perpendicular components
\pagebreak

\subsection{Errors}
\begin{defn}{Systematic error}{}
Error where repeating the measurement under the same conditions yields all measurements bigger or smaller than true value. 
\end{defn}

\begin{defn}{Random error}{}
Error where repeating the measurement under the same conditions yields all measurements scattered about mean value.
\end{defn}

\begin{table}[H]
    \centering
    \begin{tabular}{p{7.5cm}p{7.5cm}}
    \hline\hline
    \textbf{Systematic error} & \textbf{Random error} \\
    \hline
    \textbf{Same} magnitude and sign & \textbf{Different} magnitudes and signs \\
    Can be eliminated by careful design of experiment, good experimental techniques. & Cannot be eliminated, but can be reduced by repeating measurements and averaging readings by plotting a best fit line for data points. \\
    Examples: poorly calibrated instrument, instrumental zero error, human reaction time, parallax error & Examples: non-uniformity of wires, instrument sensitivity, fluctuations in the testing environment (temperature, wind), irreproducible readings (repeat timing for 20 oscillations) \\
    \hline\hline
    \end{tabular}
\end{table}

\begin{defn}{Accuracy}{}
Degree of agreement between measurements and true value. \end{defn}

\begin{defn}{Precision}{}
Degree of agreement among a series of measurements. \end{defn}

\begin{table}[H]
    \centering
    \begin{tabular}{p{7.5cm}p{7.5cm}}
    \hline\hline
    \textbf{Accuracy} & \textbf{Precision} \\
    \hline
    High accuracy is associated with small systematic error; mean value is close to true value. & High precision is associated with small random error; small scattering of readings about mean value. \\
    Graphically, line of best fit does not pass through the origin. & Graphically, data points do not lie on a straight line, but scattered around the line of best fit. \\
    \hline\hline
    \end{tabular}
\end{table}
\pagebreak

\subsection{Uncertainties}
Given a measurement $R$.
\begin{itemize}
\item \vocab{Actual uncertainty} is denoted as $\Delta R$.
\item \vocab{Fractional uncertainty} is given by $\dfrac{\Delta R}{R}$.
\item \vocab{Percentage uncertainty} is given by $\dfrac{\Delta R}{R} \times 100\%$.
\end{itemize}

When there are more quantities, uncertainty increases.

Given the measurements $R$, $A$, $B$, and coefficients $m$, $n$.
\begin{itemize}
\item For addition and subtraction where $R = mA + nB$, add or subtract \textbf{actual} uncertainties:
\begin{equation} \Delta R = |m| \Delta A + |n| \Delta B \end{equation}

\item For multiplication and division where $R = A^m B^n$, add or subtract \textbf{fractional} uncertainties:
\begin{equation} \frac{\Delta R}{R} = |m| \frac{\Delta A}{A} + |n| \frac{\Delta B}{B} \end{equation}

\item Use the \textbf{First Principle} to deal with complex expressions 
\begin{equation} \Delta R = \frac{R_{\mathrm{max}} - R_{\mathrm{min}}}{2} \end{equation}
from which we can derive 
\[ \Delta R = R_{\mathrm{max}} - R = R - R_{\mathrm{min}} \]
\end{itemize}
\pagebreak

\subsection{Problems}
\begin{prbm}
check homogeneity
\end{prbm}

\begin{prbm}
State why, by drawing a line of best fit for the data points, the effect of random error is reduced.
\end{prbm}

\begin{proof}[Answer]
Random errors have different signs and magnitudes in repeated measurements, causing readings to be scattered.

Since the line of best fit has on average an equal number of readings on both sides, errors that cause \underline{overestimation} of experimental result will partially \underline{cancel} the errors that cause \underline{underestimation}, thus reducing the effect of random errors.
\end{proof}

\begin{prbm}
Uncertainty calculation
\end{prbm}

\begin{prbm}
Precision and accuracy
\end{prbm}
\pagebreak