\section{Electrochemistry}
\subsection{Redox processes}
Oxidation: loss of electrons / increase in oxidation state

Reduction: gain of electrons / decrease in oxidation state

\subsection{Electrode potentials}
\begin{defn}{Standard electrode (redox) potential $E\stst$}{}
Potential 
\end{defn}
(i) standard electrode (redox) potentials, $E\stst$; the redox series
(ii) standard cell potentials, $E_\text{cell}\stst$, and their uses
(iii) batteries and fuel cells

\subsection{Electrolysis}
(i) factors affecting the amount of substance liberated during electrolysis
(ii) the Faraday constant; the Avogadro constant; their relationship
(iii) industrial uses of electrolysis

Learning Outcomes
Candidates should be able to:
(a) describe and explain redox processes in terms of electron transfer and/or of changes in oxidation number
(oxidation state)
(b) define the terms:
(i) standard electrode (redox) potential
(ii) standard cell potential
(c) describe the standard hydrogen electrode
(d) describe methods used to measure the standard electrode potentials of:
(i) metals or non-metals in contact with their ions in aqueous solution
(ii) ions of the same element in different oxidation states
(e) calculate a standard cell potential by combining two standard electrode potential
(f) use standard cell potentials to:
(i) explain/deduce the direction of electron flow from a simple cell
(ii) predict the spontaneity of a reaction
(g) understand the limitations in the use of standard cell potentials to predict the spontaneity of a reaction
(h) construct redox equations using the relevant half-equations (see also Section 13)
(i) state and apply the relationship $\Delta G\stst=-nFE\stst$ to electrochemical cells, including the calculation of $E\stst$

for combined half reactions
(j) predict qualitatively how the value of an electrode potential varies with the concentration of the aqueous ion
(k) state the possible advantages of developing other types of cell, e.g. the H2/O2 fuel cell and improved
batteries (as in electric vehicles) in terms of smaller size, lower mass and higher voltage
(l) state the relationship, F = Le, between the Faraday constant, the Avogadro constant and the charge on the
electron
(m) predict the identity of the substance liberated during electrolysis from the state of electrolyte (molten or
aqueous), position in the redox series (electrode potential) and concentration
(n) calculate:
(i) the quantity of charge passed during electrolysis
(ii) the mass and/or volume of substance liberated during electrolysis
(o) explain, in terms of the electrode reactions, the industrial processes of:
(i) the anodising of aluminium
(ii) the electrolytic purification of copper
\pagebreak