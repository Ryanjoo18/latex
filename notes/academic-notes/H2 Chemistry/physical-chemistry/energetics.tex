\section{Chemical Energetics}
\subsection{Thermochemistry}
\begin{defn}{Enthalpy change of reaction $\Delta H$}{}
Energy change when molar quantities of reactants as specified by the chemical equation react to form products.
\end{defn}

\begin{defn}{Hess' Law}{}
Enthalpy change accompanying a chemical reaction is the same regardless of the route by which the chemical change occurs, provided the initial and final states are the same.
\end{defn}

\begin{defn}{Standard enthalpy change of reaction $\Delta H\stst$}{}
Energy change when molar quantities of reactants as specified by the chemical equation react to form products at standard conditions.
\end{defn}

\begin{defn}{Standard enthalpy change of formation $\Delta H_f\stst$}{}
Energy change when one mole of the substance is formed from its constituent elements under standard conditions.
\[ \ce{Na (s) + 1/2 Cl2(g) -> NaCl (s)} \]
\end{defn}

\begin{equation}
\Delta H=\sum H_f\text{ (products)}-\sum H_f\text{ (reactants)}
\end{equation}

\begin{defn}{Standard enthalpy change of combustion $\Delta H_c\stst$}{}
Energy evolved when one mole of the substance is completely burnt in oxygen under standard conditions.
\[ \ce{CH4(g) + 2 O2(g) -> CO2(g) + 2 H2O(l)} \]
\end{defn}

\begin{equation}
\Delta H=\sum H_c\text{ (reactants)}-\sum H_c\text{ (products)}
\end{equation}

\begin{defn}{Standard enthalpy change of hydration $\Delta H_{\mathrm{hyd}}\stst$}{}
Energy evolved when one mole of gaseous ions is hydrated under standard conditions
Standard enthalpy change of solution.
\[ \ce{Na^+(g) -> Na^+(aq)} \]
\end{defn}

\begin{equation}
\Delta H_\text{hyd}\stst\propto\frac{q^+}{r^+}
\end{equation}

\begin{defn}{Standard enthalpy change of solution $\Delta H_{\mathrm{soln}}\stst$}{}
Energy change when one mole of substance is completely dissolved in a solvent to form an infinitely dilute solution under standard conditions.
\[ \ce{NaCl (s) -> Na^+(aq) + Cl^-(aq)} \]
\end{defn}

\begin{defn}{Standard enthalpy change of neutralisation $\Delta H_n\stst$}{}
Energy evolved when one mole of water is formed from the neutralisation between acid and base under standard conditions.
\[ \ce{H^+(aq) + OH^-(aq) -> H2O(l)} \]
\end{defn}

\begin{defn}{Standard enthalpy change of atomisation $\Delta H_{\mathrm{atom}}\stst$}{}
Energy absorbed when one mole of gaseous atoms is formed from the element under standard conditions.
\[ \ce{1/2 Cl2(g) -> Cl(g)} \]
\end{defn}

\begin{defn}{Bond dissociation energy}{}
Energy required to break one mole of covalent bond in a specific molecule in the gaseous state to form gaseous atoms.
\end{defn}

\begin{defn}{Bond energy}{}
Average energy absorbed to break one mole of covalent bond in the gaseous state to form gaseous atoms under standard conditions.
\end{defn}

\begin{equation}
\Delta H\stst=\sum\text{BE (bonds broken)}-\sum\text{BE (bonds formed)}
\end{equation}

\begin{equation}
\text{BE (\ce{A2})}=2\Delta H_\text{atom}\stst\text{ (\ce{A2})}
\end{equation}

\begin{defn}{First ionisation energy}{}
Energy absorbed when one mole of gaseous atoms loses one mole of electrons to form one mole of singly charged gaseous cations.
\[ \ce{Na(g) -> Na^+(g) + e^-} \]
\end{defn}

\begin{defn}{Second ionisation energy}{}
Energy absorbed when one mole of singly charged gaseous cations loses one mole of electrons to form one mole of doubly charged gaseous cations.
\[ \ce{Mg^+(g) -> Mg^2+(g) + e^-} \]
\end{defn}

\begin{defn}{First electron affinity}{}
Energy evolved when one mole of gaseous atoms acquires one mole of electrons to form one mole of singly charged gaseous anions.
\[ \ce{Cl(g) + e^- -> Cl^-(g)} \]
\end{defn}

\begin{defn}{Second electron affinity}{}
Energy absorbed when one mole of singly charged gaseous anions acquires one mole of electrons to form one mole of doubly charged gaseous anions.
\[ \ce{S^-(g) + e^- -> S^2-(g)} \]
\end{defn}

\begin{defn}{Lattice energy}{}
Energy evolved when one mole of the solid ionic compound is formed from its constituent gaseous ions under standard conditions.
\[ \ce{Na^+(g) + Cl^-(g) -> NaCl (s)} \]
\end{defn}

\begin{equation}
\text{LE}\propto\frac{q^+\cdot q^-}{r^++r^-}
\end{equation}

\begin{equation}
\Delta H_\text{soln}\stst=\sum\Delta H_\text{hyd}\stst-\text{LE}
\end{equation}

\begin{remark}
Take note of the following when writing equations:
\begin{enumerate}
\item State symbols
\item Stoichiometric coefficients
\item Sign for $\Delta H$
\end{enumerate}
\end{remark}

\begin{ebox}
\textbf{Values}
\begin{itemize}
\item Standard conditions: 298 K, 1 bar 
\item Standard temperature and pressure (s.t.p.): 273 K, 1 atm
\item Room temperature: $20\degree C$
\item Specific heat capacity of water: $4.18 \unit{kJ.kg.{-1}.K^{-1}}$ (or $4.18 \unit{J.g^{-1}.K^{-1}}$)
\end{itemize}
Refer to Data Booklet for bond energies, ionisation energies.
\end{ebox}

\subsubsection{Heat change}
\begin{equation}
Q=mc\Delta T
\end{equation}

\begin{equation}
\Delta H=\pm\frac{Q}{n}
\end{equation}

\subsection{Thermodynamics}
\begin{defn}{Entropy $S$}{}
Degree of disorder or randomness in a system.
\end{defn}

Factors affecting entropy change $\Delta S$:
\begin{itemize}
\item \textbf{Temperature} 

At higher temperature, average kinetic energy of particles increases. More ways to distribute greater amount of energy among particles. Entropy increases.

\item \textbf{Phase}

Particles move about more freely and with greater speeds. More ways to distribute particles and energy. Entropy increases.

\item \textbf{Number of particles}

More particles moving randomly. More ways to distribute particles and energy. Entropy increases.

\item \textbf{Expansion of volume} (gaseous system)

Larger volume. More ways to distribute particles and energy. Entropy increases.

\item \textbf{Mixing of particles}

When gases are mixed, each gas expands to occupy the whole container. More ways to distribute particles and energy in a larger volume. Entropy increases.
\end{itemize}

\subsubsection{Spontaneity}
\textbf{Gibbs free energy change} $\Delta G$ is given by
\begin{equation}
\Delta G = \Delta H - T \Delta S
\end{equation}

\begin{remark}
Note that since the units of $\Delta S$ is usually given in $\unit{J\,mol^{-1}\,K^{-1}}$, so it has to be converted to $\unit{kJ\,mol^{-1}\,K^{-1}}$ for calculations of $\Delta G$.
\end{remark}

Standard Gibbs free energy change, at standard conditions, is given by
\[ \Delta G\stst = \Delta H\stst - T \Delta S \]

The spontaneity of a reaction can be determined from the value of $\Delta G$:
\begin{table}[H]
\centering
\begin{tabular}{|c|c|}
\hline
$\Delta G < 0$ & Reaction is spontaneous \\
\hline
$\Delta G > 0$ & Reaction is non-spontaneous \\
\hline
$\Delta G = 0$ & Reaction is at equilibrium (phase change) \\
\hline
\end{tabular}
\end{table}

To determine the change in spontaneity of reaction with temperature, use the \textbf{signs} of $\Delta H$ and $-T\Delta S$ to determine change in $\Delta G$.\footnote{sketch out graph to visualise better}

\begin{table}[H]
\centering
\begin{tabular}{ccccc}
\hline\hline
$\Delta H$ & $\Delta S$ & $-T\Delta S$ & $\Delta G$ & Spontaneity \\
\hline
$-$ & $+$ & $-$ & always negative & Spontaneous at ALL temperatures \\
$+$ & $-$ & $+$ & always positive & Non-spontaneous at ALL temperatures \\
$+$ & $+$ & $-$ & negative if $|T\Delta S| > |\Delta H|$ & Spontaneous at HIGH temperatures \\
$-$ & $-$ & $+$ & negative if $|\Delta H| > |T\Delta S|$ & Spontaneous at LOW temperatures \\
\hline\hline
\end{tabular}
\end{table}





Limitations in the use of $\Delta G$ to predict spontaneity:
\begin{itemize}
\item Kinetic feasibility

Some reactions are energetically feasible (also known as thermodynamically feasible) since $\Delta G$ is negative, but kinetically not feasible since it occurs very slowly due to high activation energy. Such reactions are spontaneous but very slow.

\item Non-standard conditions

$\Delta G\stst$ can only be used to predict the spontaneity of a reaction under standard conditions. Under non-standard conditions, $\Delta G$ must be calculated.
\end{itemize}

\pagebreak

