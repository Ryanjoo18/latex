\section{Gaseous State}
\subsection{Gas Laws}
The gas laws are given by
\begin{itemize}
\item \textbf{Boyle's Law}: $p \propto \dfrac{1}{V}$ at constant $T$ and $n$
\item \textbf{Charles' Law}: $V \propto T$ at constant $p$ and $n$
\item \textbf{Gay--Lussac's Law}: $p \propto T$ at constant $V$ and $n$
\item \textbf{Avogadro's Law}: $V \propto n$ at constant $p$ and $n$
\end{itemize}

\begin{remark}
Remember to take note which quantities are \emph{variables} and \emph{constants}!
\end{remark}

\begin{remark}
SI units must be used for calculations: $p$ (in Pa), $T$ (in K), $V$ (in $\unit{m^3}$), $n$ (in mol)

Note that 1 atm = 101325 Pa, 1 bar = $10^5$ Pa, T (K) = T (\degree C) + 273
\end{remark}

\subsubsection{Ideal gas equation}
The \vocab{ideal gas equation} is given by
\begin{equation}\label{eqn:ideal_gas}
pV = nRT
\end{equation}
where molar gas constant $R = 8.31 \unit{J\,K^{-1}\,mol^{-1}}$

From \cref{eqn:ideal_gas} we can derive the expression for \textbf{molar mass} of gas:
\begin{equation}
M_r = \frac{mRT}{pV}
\end{equation}

and also the expression for \textbf{density} of gas:
\begin{equation}
\rho = \frac{pM_r}{RT}
\end{equation}

\subsubsection{Partial pressure}
\begin{defn}{Dalton's Law}{}
In a mixture of inert gases at constant volume and temperature, total pressure of mixture is the sum of partial pressures of constituent gases.
\end{defn}

\begin{equation}
p_\text{gas} = \frac{n_\text{gas}}{n_T}p_T
\end{equation}

\subsection{Kinetic theory of gases}
Basic assumptions:
\begin{enumerate}
\item Small particles of negligible volumes, as compared to container
\item Negligible intermolecular forces of attraction
\item Perfectly elastic collisions between gas particles and walls of container
\end{enumerate}

All gases are non-ideal; they are real gases.
\begin{table}[H]
\centering
\begin{tabular}{p{7.5cm}p{7.5cm}}
\hline\hline
\textbf{Approach ideality} & \textbf{Deviate from ideality} \\
\hline
High temperature: gas particles able to overcome most of the intermolecular forces of attraction & Low temperature: molecules move more slowly, intermolecular forces of attraction become less negligible \\
\hline
Low pressure: volume of gas particles becomes negligible as compared to volume occupied by gas & High pressure: intermolecular distances become less negligible \\
\hline
& Strong intermolecular forces of attraction

Large size of gas molecule \\
\hline\hline
\end{tabular}
\end{table}

\begin{defn}{Compressibility}{}
Ratio of measured molar volume $V_m$ to molar volume of ideal gas $V_m^\circ$ at same temperature and pressure.
\begin{equation}
Z = \frac{V_m}{V_m^\circ} = \frac{pV_m}{RT}
\end{equation}
\end{defn}

\subsubsection{Compressibility against pressure}
[graph]

At low pressure, $Z<1$. Reason: Attractive forces between molecules, molar volume $V_m$ is smaller than that of ideal gas $V_m^\circ$.

At high pressure, $Z>1$. Reason: Repulsive forces between molecules, molar volume $V_m$ is larger than that of ideal gas $V_m^\circ$.

\subsubsection{Compressibility against temperature}
[graph]

Temperature decreases, deviation from ideality increases
Average kinetic energy of gas particles decreases
Gas particles closer together, intermolecular forces of attraction become significant 
$V_m$ smaller than that of ideal gas

\pagebreak

