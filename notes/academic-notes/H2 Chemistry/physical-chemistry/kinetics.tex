\section{Reaction Kinetics}
\begin{defn}{Rate of reaction}{}
Rate of change in concentration of reactant or product.
\begin{equation}
\text{rate of reaction} = \dv{[\text{product}]}{t} = -\dv{[\text{reactant}]}{t}
\end{equation}
\end{defn}

\subsection{Rate of reaction}
\subsubsection{Theories}
\begin{defn}{Collision Theory}{}
Chemical reactions occur only if reactant particles undergo \vocab{effective collisions}, where reactant particles collide
\begin{enumerate}
\item with energy greater than or equal to activation energy
\item in the correct orientation
\end{enumerate}
\end{defn}

\begin{defn}{Transition State Theory}{}
When reactant molecules collide, a \vocab{transition state} is formed, whereby old bonds are \emph{partially} broken and new bonds are \emph{partially} formed.

Transition state has maximum energy in reaction pathway (peak of curve) and is very unstable, thus cannot be isolated as a compound.
\end{defn}

A \vocab{Maxwell-Boltzmann distribution curve}: represents amount of kinetic energy possessed by a particular fraction of particles.

\subsubsection{Factors affecting rate of reaction}
\begin{table}[H]
\begin{tabular}{|p{4cm}|p{11cm}|}
\hline
\textbf{Factor} & \textbf{Explanation} \\
\hline
\textbf{Temperature} & Average kinetic energy of reactant particles increases.

Number of reactant particles with energy $\ge E_a$ increases. 

Frequency of effective collisions increases. Since rate of reaction is proportional to frequency of effective collisions, rate of reaction increases.

Distribution curve: displaced towards the right, peak is lowered \\
\hline
\textbf{Catalyst} & In presence of a catalyst, activation energy is lowered. 

Number of reactant particles with energy $\ge E_a$ increases. 

Frequency of effective collisions increases. Since rate of reaction is proportional to frequency of effective collisions, rate of reaction increases.

Distribution curve: $E_a^\prime$ (catalysed) is to the left of $E_a$ (uncatalysed) \\
\hline
\textbf{Concentration of reactant} & Number of reactant particles per unit volume increases. 

Frequency of effective collisions increases. Since rate of reaction is proportional to frequency of effective collisions, rate of reaction increases. \\
\hline
\textbf{Physical state of reactant} & Surface area over which solid comes into contact with liquid or gaseous reactants is larger. 

Frequency of effective collisions increases. Since rate of reaction is proportional to frequency of effective collisions, rate of reaction increases. \\
\hline
\textbf{Light} & Upon absorbing light energy, average kinetic energy of reactant particles increases. 

Number of reactant particles with energy $\ge E_a$ increases. 

Frequency of effective collisions increases. Since rate of reaction is proportional to frequency of effective collisions, rate of reaction increases. \\
\hline
\end{tabular}
\end{table}

\subsection{Order of reaction}


\begin{defn}{Rate equation}{}
A mathematical equation that relates rate of reaction to concentration of reactants raised to appropriate powers.
\begin{equation}
\text{rate} = k[A]^m[B]^n
\end{equation}
where $m$ and $n$ are \textbf{orders of reaction} with respect to reactants A and B respectively, i.e. the powers to which the concentration of reactants are raised to in the rate equation (determined experimentally).

$k$ is the \textbf{rate constant}, i.e. constant of proportionality in the rate equation. It is constant for a given reaction at a \emph{particular temperature}.
\end{defn}

\begin{remark}
The rate constant can be determined using the \textbf{Arrhenius equation}, which is useful to know but not essential.
\begin{equation}
k = Ae^{-\frac{E_a}{RT}}
\end{equation}
This shows that rate constant is only affected by temperature and activation energy.
\end{remark}

\begin{defn}{Half-life}{}
Time taken for concentration of reactant to decrease to half of its original value.
\end{defn}

Half life and rate constant are related by the following equation.
\begin{equation}
t_\frac{1}{2} = \frac{\ln 2}{k}
\end{equation}

\subsection{Experimental techniques}

\subsection{Reaction mechanism}
\begin{defn}{Rate-determining step}{}
Slowest step in the sequence of steps leading to formation of product.
\end{defn}

A reaction consist of \textbf{one slow step} and \textbf{many fast steps}.

Slow step has high activation energy (e.g. strong bonds to break), is rate-determining step

rate equation for overall equation is obtained from stoichiometry of rate-determining step (stoichiometric coefficient of reactant = order of reaction wrt reactant)
\begin{remark}
If slow step contains intermediates, look at fast step for reactants.
\end{remark}

\subsection{Catalysis}

\begin{defn}{Activation energy}{}
Minimum amount of energy that reactant particles must possess for effective collisions to result in chemical reaction.
\end{defn}

\begin{defn}{Catalyst}{}
Substance that increases rate of reaction by providing an alternative pathway with lower activation energy while remaining chemically unchanged.
\end{defn}



\pagebreak

