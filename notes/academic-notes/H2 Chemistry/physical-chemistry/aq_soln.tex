\section{Chemistry of Aqueous Solutions}
\subsection{Theories of Acids and Bases}
\begin{table}[H]
\centering
\begin{tabular}{p{2cm}p{6.5cm}p{6.5cm}}
\hline\hline
Theory & \textbf{Acid} & \textbf{Base} \\
\hline
\vocab{Arrhenius} & hydrogen-containing substance, ionises and releases \ce{H+} ions in aq solution & hydroxide-containing substance, ionises and releases \ce{OH-} ions in aq solution \\
\vocab{Br{\o}nsted--Lowry} & proton donor & proton acceptor \\
\vocab{Lewis} & electron pair acceptor (electrophile) & electron pair donor (nucleophile) \\
\hline\hline
\end{tabular}
\end{table}

\begin{remark}
Let $A$ denote the set of Arrhenius acids and bases, $B$ denote the set of Br{\o}nsted--Lowry acids and bases, $L$ denote the set of Lewis acids and bases. Then
\[ A \subset B \subset L. \]
\end{remark}

\subsubsection{Conjugate acid-base pair}
\vocab{Conjugate acid-base pair}: a pair of compounds or ions which \emph{differ by one proton}
\begin{itemize}
\item When acid \ce{HA} donates proton to base, conjugate base \ce{A-} is formed.
\[ \ce{HA <=> A- + H+} \]
\item When base \ce{B} receives proton from acid, conjugate acid \ce{BH+} is formed.
\[ \ce{B + H+ <=> BH+} \]
\end{itemize}
Conjugate acid-base pairs: \ce{HA} and \ce{A-}, \ce{B} and \ce{BH+}

\subsubsection{Lewis acids}
Types of species which can act as Lewis acids
\begin{itemize}
\item Electron-deficient molecules

Less than an octet of electrons around one atom. Dative bond formed in adduct\footnote{compound that contains a dative bond between the Lewis acid and the Lewis base}.

\item Molecules with central atom that can expand octet

Low-lying vacant orbitals to accept lone pairs of electrons.

\item Molecules with multiple bonds that has atom with partial positive charge

Lewis base donates lone pair of electrons to form bond with central atom, one pair of electrons displaced from multiple bond to become lone pair on terminal atom.

\item Metal cations

Accept lone pair of electrons via dative bond to form complex ions.
\end{itemize}
\pagebreak

\subsection{Acid--Base Equilibria}
\subsubsection{Strength of acids and bases}
\vocab{Strength} is a measure of extent of dissociation to give ions in solution. (independent of concentration)

\begin{table}[H]
\centering
\begin{tabular}{p{4cm}p{5.5cm}p{5.5cm}}
\hline\hline
 & \vocab{Strong} & \vocab{Weak} \\
\hline
Definition & Complete dissociation in aq solution & Partial dissociation in aq solution \\
Extent of dissociation & $100\%$ & $<<100\%$ \\
Arrow & \ce{->} & \ce{<=>} \\
Equilibrium constant & -- & $K_a$, $K_b$ \\
\hline\hline
\end{tabular}
\end{table}

\subsubsection{Acid and base dissociation constants}
\vocab{Power of hydrogen}:
\begin{equation}
pH = -\lg [\ce{H+}]
\end{equation}

\vocab{Power of hydroxide}:
\begin{equation}
pOH = -\lg [\ce{OH-}]
\end{equation}

\vocab{Acid dissociation constant}: measure of strength of \emph{weak} acid\footnote{equilibrium constant}
\begin{equation}
K_a = \frac{[\ce{H+}][\ce{A-}]}{[\ce{HA}]}
\end{equation}

\begin{equation}
pK_a = -\lg K_a
\end{equation}

\vocab{Base dissociation constant}: measure of strength of \emph{weak} base
\begin{equation}
K_b = \frac{[\ce{BH+}][\ce{OH-}]}{[\ce{B}]}
\end{equation}

\begin{equation}
pK_b = -\lg K_b
\end{equation}

Acidic, basic \& neutral solutions
\begin{table}[H]
\centering
\begin{tabular}{cc}
\hline\hline
Solution & Meaning \\
\hline
\textbf{acidic} & $[\ce{H+}]>[\ce{OH-}]$ \\
\textbf{basic} & $[\ce{H+}]<[\ce{OH-}]$ \\
\textbf{neutral} & $[\ce{H+}]=[\ce{OH-}]$ \\
\hline\hline
\end{tabular}
\end{table}

\subsubsection{Ionic product of water}
Self-ionisation of water can be simplified to \ce{H2O(l) <=> H+(aq) + OH-(aq)}.

\vocab{Ionic product of water}:
\begin{equation}
K_w = [\ce{H+}][\ce{OH-}] = 1.0\times10^{-14}\,\unit{mol^2\,dm^{-6}} \text{ at 298 K}
\end{equation}

By manipulation we have
\begin{equation}
pH + pOH = pK_w = 14 \text{ at 298 K}
\end{equation}

The following relationship holds for a conjugate acid-base pair:
\begin{equation}
K_w = K_a K_b
\end{equation}

This suggests a reciprocal strength relationship: the stronger the acid, the weaker its conjugate base (and vice versa).

\subsubsection{Salt solutions}
Salts undergo hydration to form solutions. Then cations / anions undergo hydrolysis to form acidic / alkaline solutions.

\begin{table}[H]
\centering
\begin{tabular}{p{7.5cm}p{7.5cm}}
\hline\hline
\vocab{Acidic salt solution} & \vocab{Alkaline salt solution} \\
\hline
cation: conjugate acid of WB hydrolyses partially in water to give \ce{H3O+} & anion: conjugate base of WA hydrolyses partially in water to give \ce{OH-} \\
\ce{BH+(aq) + H2O(l) <=> B(aq) + H3O+(aq)} & \ce{A-(aq) + H2O(l) <=> HA(aq) + OH-(aq)} \\
\hline\hline
\end{tabular}
\end{table}

\subsubsection{Buffer solutions}
\begin{defn}{Buffer solution}{}
A solution capable of maintaining a fairly constant pH (by resisting pH change) when small amounts of acid or base are added to it.
\end{defn}

\begin{table}[H]
\centering
\begin{tabular}{p{7.5cm}p{7.5cm}}
\hline\hline
\vocab{Acidic buffer solution} & \vocab{Basic buffer solution} \\
\hline
weak acid + salt of conjugate base (\ce{HA} and \ce{A-}) & weak base + salt of conjugate acid (\ce{B} and \ce{BH+}) \\
\hline
On addition of acid:

\ce{A-(aq) + H+(aq) -> HA(aq)} 
& On addition of acid:

\ce{B(aq) + H+(aq) -> BH+(aq)} \\
On addition of base: 

\ce{HA(aq) + OH-(aq) -> A-(aq) + H2O(l)}
& On addition of base:

\ce{BH+(aq) + OH-(aq) -> B(aq) + H2O(l)} \\
\hline
Large reservoir of \ce{HA} and \ce{A-} present is able to cope with small amount of \ce{H+} and \ce{OH-} added. & Large reservoir of \ce{B} and \ce{BH+} present is able to cope with small amount of \ce{H+} and \ce{OH-} added. \\
\hline\hline
\end{tabular}
\end{table}

Buffer solutions are used in systems where pH must not deviate widely. \ce{H2CO3} / \ce{HCO3-} buffer pair is used to control pH of blood:
\[ \ce{H2CO3(aq) <=> H+(aq) + HCO3-(aq)} \]
\begin{itemize}
\item On addition of acid: \ce{HCO3-(aq) + H+(aq) -> H2CO3(aq)}
\item On addition of base: \ce{H2CO3(aq) + OH-(aq) -> HCO3-(aq) + H2O(l)}
\end{itemize}
\pagebreak

\subsubsection{Titration}
Titrate titrant against analyte / Analyte titrated against titrant

\vocab{Equivalence point}: stoichiometric amounts of acid \& base react together

\vocab{End-point}: when indicator first changes colour permanently

Choice of pH indicator is considered appropriate if its pH transition range lies within range of rapid pH change over equivalence point.

pH graph sketching
\begin{enumerate}
\item initial, equivalence, final pH
\item initial, equivalence, final volume
\item buffer region
\end{enumerate}

Types of titration
\begin{itemize}
\item strong acid--strong base

\item strong acid--weak base

\item weak acid--strong base

\item weak acid--weak base
\end{itemize}

(d) describe the changes in pH during acid-base titrations and explain these changes in terms of the strengths of the acids and bases
(e) explain the choice of suitable indicators for acid-base titrations, given appropriate data

\subsubsection{Calculations}
\begin{itemize}
\item \textbf{Water}
\[ \ce{H2O(l) <=> H+(aq) + OH-(aq)} \]
In pure water, $[\ce{H+}]=[\ce{OH-}]$

$K_w=[\ce{H+}][\ce{OH-}]=10^{-14}\,\unit{mol^2\,dm^{-6}}$

$[\ce{H+}]=[\ce{OH-}]=10^{-7}\,\unit{mol\,dm^{-3}}$

\item \textbf{Strong acid}
\[ \ce{H_nA(aq) -> nH+(aq) + A-(aq)} \]
$[\ce{H+}]=n\times[\ce{H_nA}]$

\vspace{5mm}

\textbf{Strong base}
\[ \ce{M(OH)_n(aq) -> M^{n+}(aq) + nOH-(aq)} \]
$[\ce{OH-}]=n\times[\ce{M(OH)_n}]$

\item \textbf{Weak acid} (monobasic)
\[ \ce{HA(aq) <=> A-(aq) + H+(aq)} \]
Since \ce{HA} is weak acid with small $K_a$, assume extent of dissociation of \ce{HA} is negligible, so $[\ce{H+}]$ is so small such that $[\ce{HA}]\approx[\ce{HA}]_\text{initial}$.

$\displaystyle K_a=\frac{[\ce{H+}][\ce{A-}]}{[\ce{HA}]}=\frac{[\ce{H+}]^2}{[\ce{HA}]}\approx\frac{[\ce{H+}]^2}{[\ce{HA}]_\text{initial}}$

\[ \boxed{[\ce{H+}]=\sqrt{K_a\times[\ce{HA}]_\text{initial}}} \]
from which we can calculate $pH$.

\begin{remark}
This formula CANNOT be used to calculate $K_a$ or $[\ce{HA}]_\text{initial}$! Instead use ICE table.
\end{remark}

\vspace{5mm}

\textbf{Weak base} (monoacidic)
\[ \ce{B(aq) + H2O(l) <=> BH+(aq) + OH-(aq)} \]
Since \ce{B} is weak base with small $K_b$, assume extent of dissociation of \ce{B} is negligible, so $[\ce{OH-}]$ is so small such that $[\ce{B}]=[\ce{B}]_\text{initial}$.

$\displaystyle K_b=\frac{[\ce{BH+}][\ce{OH-}]}{[\ce{B}]}=\frac{[\ce{OH-}]^2}{[\ce{B}]}\approx\frac{[\ce{OH-}]^2}{[\ce{B}]_\text{initial}}$
\[ \boxed{[\ce{OH-}]=\sqrt{K_b\times[\ce{B}]_\text{initial}}} \]
from which we can calculate $pOH$ and then $pH$.

\begin{remark}
This formula CANNOT be used to calculate $K_b$ or $[\ce{B}]_\text{initial}$! Instead use ICE table.
\end{remark}

\item \textbf{Acidic salt solution}
\[ \ce{BH+X-(aq) -> BH+(aq) + X-(aq)} \]
\[ \ce{BH+(aq) + H2O(l) <=> B(aq) + H3O+(aq)} \]

\[ \boxed{[\ce{H+}]=\sqrt{K_a\times[\ce{HA}]_\text{initial}}} \]

\vspace{5mm}

\textbf{Alkaline salt solution}
\[ \ce{M+A-(aq) -> M+(aq) + A-(aq)} \]
\[ \ce{A-(aq) + H2O(l) <=> HA(aq) + OH-(aq)} \]

\[ \boxed{[\ce{OH-}]=\sqrt{K_b\times[\ce{B}]_\text{initial}}} \]

\begin{remark}
When solutions are mixed, remember to calculate the new concentrations.
\end{remark}

\item \textbf{Acidic buffer solution}
\[ \boxed{pH = pK_a + \lg\frac{[\ce{A-}]}{[\ce{HA}]}} \]
Maximum buffer capacity: $[\ce{HA}]=[\ce{A-}] \implies pH=pK_a$

\vspace{5mm}

\textbf{Basic buffer solution}
\[ \boxed{pOH = pK_b + \lg\frac{[\ce{BH+}]}{[\ce{B}]}} \]
Maximum buffer capacity: $[\ce{B}]=[\ce{BH+}] \implies pOH=pK_b$

\end{itemize}

Taking -log on both sides gives us
\begin{equation}
pK_a + pK_b = pK_w
\end{equation}
At 298 K, $pK_a+pK_b=14$.

(c) calculate [H+(aq)] and pH values for strong acids, weak monobasic (monoprotic) acids, strong bases, and weak monoacidic bases [Calculations involving weak acids/bases will not require solving of quadratic equations]


\pagebreak

\subsection{Solubility Equilibria}
\begin{defn}{Solubility}{}
\emph{Maximum} mass / amount of solute that can be dissolved per $\unit{dm^3}$ of solvent to produce a \emph{saturated solution} at a given temperature.
\end{defn}

\begin{defn}{Solubility product $K_{sp}$}{}
\emph{Equilibrium constant} which is the product of molar concentrations of dissolved / dissociated ions (each raised to its appropriate power) in a \emph{saturated solution} of salt at a given temperature.
\end{defn}

For a sparingly soluble salt \ce{MX}, \ce{MX (s) <=> M+(aq) + X-(aq)}.
\begin{equation}
K_{sp} = [\ce{M+}]_\text{satn}[\ce{X-}]_\text{satn}
\end{equation}

Factors affecting solubility of salt
\begin{itemize}
\item \textbf{Common ion effect} (addition of cation/anion) 

Reduced solubility of a solute in a solution that already contains the same ion.

Saturated sparingly soluble salt solution \ce{MX} added to soluble salt solution \ce{NaX(aq)} (containing common ion \ce{X-}). $[\ce{X-}]$ increases. By Le Chatelier's Principle, position of equilibrium shifts left to decrease $[\ce{X-}]$, so dissociation of \ce{MX} supressed, \ce{MX} precipitated. Hence solubility decreases in \ce{NaX(aq)} compared to that in water.

\item \textbf{Formation of complex ions} (removal of cation)

Suitable base added to saturated sparingly soluble salt solution \ce{MX}. Formation of complex ions remove cations \ce{M+}, $[\ce{M+}]$ decreases. By Le Chatelier's Principle, position of equilibrium shifts right to increase $[\ce{M+}]$, so dissociation of \ce{MX} favoured. Hence solubility increases.

$[\ce{M+}]$ decreases, ionic product decreases to below $K_{sp}$, so \ce{MX} dissolves.

\item \textbf{pH of solution} (removal of anion)

\ce{H+} added to saturated sparingly soluble salt solution \ce{MOH}. $[\ce{OH-}]$ decreases. By Le Chatelier's Principle, position of equilibrium shifts right to increase $[\ce{OH-}]$. Hence solubility increases.
\end{itemize}

\begin{defn}{Ionic product $\text{IP}$}{}
Product of molar concentrations of constituent ions in solution \emph{at that instant} (each raised to its appropriate power) at a given temperature.
\end{defn}
\begin{equation}
\text{IP} = [\ce{M+}][\ce{X-}]
\end{equation}

Precipitation occurs when $\text{IP}>K_{sp}$.
\begin{table}[H]
\centering
\begin{tabular}{ccc}
\hline\hline
\textbf{Condition} & \textbf{Saturation} & \textbf{Precipitation} \\
\hline
$\text{IP}<K_{sp}$ & Below saturation point (unsaturated) & No \\
$\text{IP}=K_{sp}$ & At saturation point (saturated) & No \\
$\text{IP}>K_{sp}$ & Beyond saturation point (beyond saturation) & Yes \\
\hline\hline
\end{tabular}
\end{table}


Candidates should be able to:
(b) calculate Ksp from concentrations and vice versa
(c) discuss the effects on the solubility of ionic salts by the following:
(i) common ion effect
(ii) formation of complex ion, as exemplified by the reactions of halide ions with aqueous silver ions followed by aqueous ammonia (see also Section 13)

\pagebreak