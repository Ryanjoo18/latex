\section{Transition Elements}
\subsection{Physical properties}
\subsection{Chemical properties}
\subsection{Colour of complexes}

(a) explain what is meant by a transition element, in terms of d block elements forming one or more stable ions with partially filled d subshells
(b) state the electronic configuration of a first row transition element and its ions
(c) explain why atomic radii and first ionisation energies of the transition elements are relatively invariant
(d) contrast, qualitatively, the melting point and density of the transition elements with those of calcium as a
typical s block element
(e) describe the tendency of transition elements to have variable oxidation states
(f) predict from a given electronic configuration, the likely oxidation states of a transition element
(g) describe and explain the use of Fe3+/Fe2+, MnO4
-/Mn2+ and Cr2O7
2-/Cr3+ as examples of redox systems
(see also Section 12)
(h) predict, using $E\stst$ values, the likelihood of redox reactions
(i) define the terms ligand and complex as exemplified by the complexes of copper(II) ions with water,
ammonia and chloride ions as ligands
(including the transition metal complexes found in the Qualitative Analysis Notes)
(j) explain qualitatively that ligand exchange may occur, as exemplified by the formation of the complexes in
(i), including the colour changes involved, and CO/O2 exchange in haemoglobin
(k) describe, using the shape and orientation of the d orbitals, the splitting of degenerate d orbitals into two
energy levels in octahedral complexes
(l) explain, in terms of d orbital splitting and d-d transition, why transition element complexes are usually
coloured
[knowledge of the relative order of ligand field strength is not required]
(m) explain how some transition elements and/or their compounds can act as catalysts (see also 8(j)) 