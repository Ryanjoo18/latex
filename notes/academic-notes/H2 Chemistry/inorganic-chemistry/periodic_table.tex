\section{Periodic Table}
\subsection{Period 3 Elements}
\subsubsection{Reaction of oxides with water}
\[ \ce{Na2O + H2O -> 2 NaOH} \]
\[ \ce{MgO + H2O <=> Mg(OH)2} \]
\[ \ce{P4O10 + 6 H2O -> 4 H3PO4} \]
\[ \ce{SO3 + H2O -> H2SO4} \]

\subsubsection{Reaction of chlorides with water}
\[ \ce{NaCl -> Na^+ + Cl^-} \]
\[ \ce{MgCl2 + 6 H2O -> [Mg(H2O)6]^2+ + 2 Cl^-} \]
\[ \ce{[Mg(H2O)6]^2+ + H2O <=> [Mg(H2O)5(OH)]^+ + H3O^+} \]
\[ \ce{AlCl3 + 6 H2O -> [Al(H2O)6]^3+ + 3 Cl^-} \]
\[ \ce{[Al(H2O)6]^3+ + H2O <=> [Al(H2O)5(OH)]^2+ + H3O^+} \]
\[ \ce{SiCl4 + 2 H2O -> SiO2 + 4 HCl} \]
\[ \ce{PCl5 + 4 H2O -> H3PO4 + 5 HCl} \tab \text{in excess water} \]
\[ \ce{PCl5 + H2O -> POCl3 + 2 HCl} \tab \text{in limited water} \]
\[ \ce{POCl3 + 3 H2O -> H3PO4 + 3 HCl} \tab \text{when more water is added} \]

\subsubsection{Acid/base behaviour of oxides}
\[ \ce{Na2O + 2 HCl -> 2 NaCl + H2O} \]
\[ \ce{MgO + 2 HCl -> MgCl2 + H2O} \]
\[ \ce{Al2O3 + 6 HCl -> 2 AlCl3 + 3 H2O} \]
\[ \ce{Al2O3 + 2 NaOH + 3 H2O -> 2 NaAl(OH)4} \]
\[ \ce{SiO2 + 2 NaOH -> Na2SiO3 + H2O} \]
\[ \ce{P4O10 + 12 NaOH -> 4 Na3PO4 + 6 H2O} \]
\[ \ce{SO3 + 2 NaOH -> Na2SO4 + H2O} \]

\subsubsection{Acid/base behaviour of hydroxides}
\[ \ce{NaOH + HCl -> NaCl + H2O} \]
\[ \ce{Mg(OH)2 + 2 HCl -> MgCl2 + 2 H2O} \]
\[ \ce{Al(OH)3 + 3 HCl -> AlCl3 + 3 H2O} \]
\[ \ce{Al(OH)3 + NaOH -> NaAl(OH)4} \]
\pagebreak

\subsection{Group 2 Elements}
\subsubsection{Reducing agents}
Reactivity as reducing agents increases down the group
\begin{itemize}
\item \textbf{Ionisation energies, ease of losing electrons}

Down the group, number of electronic shells increases so screening effect increases, and each successive element has valence electrons located in shell with higher principal quantum number.

Valence electrons are increasingly further away from nucleus, less strongly attracted to nucleus, outweigh increase in nuclear charge.

Less energy required to move valence electron. Sum of first and second ionisation energy of Group 2 metals decrease down the group. Increase ease of atom losing electrons to form cations during oxidation. Reducing power (and hence reactivity) of Group 2 metals increases.

\item \textbf{$E\stst$ values, ease of losing electrons}

Standard electrode potential $E\stst$ value is a measure of tendency for species $\mathrm{M}^{n+}$ to undergo reduction.
\[ \ce{M^{n+} + ne^- <=> M} \]
$E\stst$ value more positive, greater tendency for $\mathrm{M}^{n+}$ to be reduced; $E\stst$ value more negative, greater tendency for M to be oxidised to $\mathrm{M}^{n+}$.

\item \textbf{Reaction of metal with oxygen}

\textbf{Reaction of metal with water}
\end{itemize}

\subsubsection{Thermal stability of compounds}
Down the group,
\begin{itemize}
\item ionic radius of cation increases, charge density of cation decreases (since charge remains the same), polarising power of cation decreases
\item extent of polarisation of electron cloud of \ch{CO3^{2-}} decreases, extent of weakening of C--O bond within \ch{CO3^{2-}} decreases
\item thermal stability of carbonates increases, hence decomposition temperature increases
\end{itemize}

\begin{exercise}{}{}
Why does \ch{Mg(OH)2} decompose at a lower temperature than \ch{Ca(OH)2}?\end{exercise}
\begin{proof}[Answer]
\ch{Mg^{2+}} ion being smaller in size than \ch{Ca^{2+}} ion has a greater charge density and hence greater polarising power. The \ch{OH-} ion is thus more polarised by the smaller \ch{Mg^{2+}} ion, O--H bond in \ch{Mg(OH)2} is weakened to larger extent. Hence \ch{Mg(OH)2} is less stable than \ch{Ca(OH)2} and thus decomposes at lower temperature.
\end{proof}
\pagebreak

\subsection{Group 17 Elements}
Trends and variations in atomic and physical properties
For elements in the third period (sodium to chlorine), and in Group 2 (magnesium to barium) and Group 17
(chlorine to iodine) candidates should be able to:
(a) recognise variation in the electronic configurations across a Period and down a Group
(b) describe and explain qualitatively the general trends and variations in atomic radius, ionic radius, first
ionisation energy and electronegativity:
(i) across a Period in terms of shielding and nuclear charge
(ii) down a Group in terms of increasing number of electronic shells, shielding and nuclear charge
(c) interpret the variation in melting point and in electrical conductivity across a Period in terms of structure and
bonding in the elements (metallic, giant molecular, or simple molecular)
(d) describe and explain the trend in volatility of the Group 17 elements in terms of instantaneous dipoleinduced dipole attraction 
\pagebreak

