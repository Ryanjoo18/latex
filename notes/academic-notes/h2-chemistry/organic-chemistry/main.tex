\documentclass[12pt,a4 paper]{article}
\usepackage{ryan}
\usepackage{appendix}
\pgfplotsset{compat=newest}
\renewcommand*{\arraystretch}{1.5}

\begin{document}
\title{\textbf{H2 Chemistry\\ Organic Chemistry}}
\author{Ryan Joo}

\maketitle

\begin{abstract}
This set of notes is written with the intention to provide readers with a brief summary of each topic in the Singapore GCE A-Level Chemistry at the H2 Level. The full syllabus can be found \href{https://www.seab.gov.sg/docs/default-source/national-examinations/syllabus/alevel/2024syllabus/9729_y24_sy.pdf}{here}.

As the notes are rather terse, you are recommended to use the lecture notes provided by your school for the understanding of more fundamental concepts.
\end{abstract}
\pagebreak

\tableofcontents
\pagebreak

\section*{Question Types}
\begin{enumerate}
\item \textbf{Mechanism question}

Remember to indicate: arrows, dipole moments, lone pairs

\item \textbf{Synthesis question}
	\begin{itemize}
	\item Change in functional group
	
	Method 1: Change the functional group by substitution. If the functional group cannot be substituted to give the desired compound in a single step, substitute the functional group into another functional group that can undergo substitution readily (e.g.
	halogenoalkane)
	
	Method 2: Convert the functional group by oxidation/reduction if the starting compound cannot undergo substitution.
	
	\item Change in position of functional group (may or may not be the same as starting reagent)
	
	Step 1: Convert the starting reagent into an alkene by elimination of the functional group.
	
	Step 2: Add the new functional group (same or different) back to the alkene.
	
	\item Increase in carbon chain length
	
	Method 1: Electrophilic substitution of nitrile group onto compound with halogen group. Nitrile group can be further converted into amine or carboxylic acid group.
	[This method does not increase the number of functional groups in the compound.]
	
	Method 2: Electrophilic Add the nitrile group into the compound using HCN. (i.e. The compound needs to have a carbonyl group already) The product formed will contain a hydroxyl group adjacent to the nitrile group. The nitrile group can be further converted into amine or carboxylic acid functional group.
	[This method increases the number of functional groups in the compound by 1.]
	
	Method 3: Employ other reactions that will increase in number of carbon atoms by more than one. (e.g. Reaction of alcohol with carboxylic acid, phenol with acyl chloride, amine with acyl chloride)
	
	\item Decrease in carbon chain length
	
	Method 1: Oxidise the alkene to rupture the \ce{C=C} double bond.
	
	Method 2: Oxidise the compound that contain \ce{CH3CH(OH){-}} group or the \ce{CH3CO{-}} group using alkaline \ce{I2(aq)}. This reduces the number of carbon atoms by one.
	
	Method 3: Hydrolysis of esters and amides. 
	\end{itemize}

\item \textbf{Structural elucidation question}



\item \textbf{Distinguishing test}

\item \textbf{Explanation question}

\end{enumerate}
\pagebreak

\section{Introduction}
\subsection{Formulae}
\begin{table}[H]
\begin{tabular}{p{0.35\textwidth}p{0.55\textwidth}}
\hline\hline
Formula & Description \\
\hline
\vocab{Empirical formula} & simplest ratio of number of atoms of the elements \\
\vocab{Molecular formula} & actual number of atoms of the elements present \\
\vocab{Structural formula} & atoms sequentially arranged, using conventional groups for an unambiguous structure \\
\vocab{Displayed formula} (full structural formula) & detailed structure of all atoms and bonds, with relative placing of atoms \\
\vocab{Skeletal formula} & only show bonds and functional groups \\
\vocab{Stereochemical formula} & 3D spatial arrangement \\
\hline\hline
\end{tabular}
\end{table}

\subsection{Functional Groups}
Classes of compounds
\begin{enumerate}
\item hydrocarbons (alkanes, alkenes, arenes)
\item halogen derivatives (halogenoalkanes, halogenoarenes)
\item hydroxyl compounds (alcohols, phenols)
\item carbonyl compounds (aldehydes, ketones)
\item carboxylic acids and derivatives (acyl chlorides, esters)
\item nitrogen compounds (amines, amides, amino acids, nitriles)
\end{enumerate}

\begin{defn}{Functional group}{}
An atom or a group of atoms common to a series, governs the principal chemical reaction of the series
\end{defn}

\begin{longtable}[H]{p{5cm}p{4cm}p{6cm}}
\toprule
\textbf{Class of compound} & \textbf{General formula} & \textbf{Functional group} \\
\midrule
\endhead
Alkane & \ce{C_{n}H_{2n+2}} & \\*
\midrule
Alkene & \ce{C_{n}H_{2n}} & \chemfig{C(-[:120])(-[:-120])=C(-[:60])-[:-60]} \\*
\midrule
Alkyne & \ce{C_{n}H_{2n-2}} & \chemfig{-C~C-} \\*
\midrule
Benzene & \ce{C6H6} & \chemfig{[:-30]**6(------)} \\*
\midrule
Halogenoalkane & \ce{C_{n}H_{2n+1}X} & \chemfig{-X} \\*
\midrule
Halogenoarene & \ce{C6H5X} & \chemfig{[:-30]**6(---(-X)---)} \\*
\midrule
Alcohol & \ce{ROH} & \chemfig{-OH} \\*
\midrule
Ether & \ce{ROR{'}} & \chemfig{-C(-[2])(-[6])-O-C(-[2])(-[6])-} \\*
\midrule
Aldehyde & \ce{RCHO} & \chemfig{-C(=[2]O)-H} \\*
\midrule
Ketone & \ce{RCOR{'}} & \chemfig{-C(=[2]O)-} \\*
\midrule
Carboxylic acid & \ce{RCOOH} & \chemfig{-C(=[:60]O)-[:-60]OH} \\*
\midrule
Acyl chloride & \ce{RCOCl} & \chemfig{-C(=[:60]O)-[:-60]Cl} \\*
\midrule
Ester & \ce{RCOOR{'}} & \chemfig{-C(=[2]O)-O-} \\*
\midrule
Amine & \ce{RNH2} & \chemfig{-NH_2} \\*
\midrule
Amide & \ce{RCONH2} & \chemfig{-C(=[2]O)-N(-[6]H)-H} \\*
\midrule
Amino acid & \ce{H2NC_{n}H_{2n}COOH} & \chemfig{H_2N-C_nH_{2n}-C(=[:60]O)-[:-60]OH} \\*
\midrule
Nitrile & \ce{RCN} & \chemfig{-C~N} \\*
\bottomrule
\end{longtable}

\begin{defn}{Homologous series}{}
A family of organic compounds with similar general formula.
\end{defn}

Successive members are represented by a general formula, differ by constant \ce{CH2} units in the carbon skeleton
\begin{itemize}
\item Physical properties: graduation in physical properties (increase in molecular size and mass)
\item Chemical properties: similar chemical properties (same functional group)
\end{itemize}
\pagebreak

\subsection{Nomenclature}

\pagebreak

\subsection{Terminology for reactions}
Types of organic species
\begin{table}[H]
\centering
\begin{tabular}{p{0.15\textwidth}p{0.15\textwidth}p{0.3\textwidth}p{0.35\textwidth}}
\hline\hline
\textbf{Species} & \textbf{Notation} & \textbf{Definition} & \textbf{Description} \\
\hline
\vocab{free radical} & Cl\textbullet & Highly reactive, electrically neutral, has unpaired electron & Attack atom in a molecule to form new bond and generate another radical \\
\vocab{nucleophile} (Lewis base) & :Nu$^\ominus$ & Electron pair donor & Electron-rich (negatively charged ion, lone pair, $\delta-$ in polar bonds)

Attracted to electron-deficient sites \\
\vocab{electrophile} (Lewis acid) & E$^\oplus$ & Electron pair acceptor & Electron-deficient (positively charged ion, $\delta+$ in polar bonds)

Attracted to electron-rich sites  \\
\hline\hline
\end{tabular}
\end{table}

Types of bond fission
\begin{table}[H]
\begin{tabular}{p{0.5\textwidth}p{0.5\textwidth}}
\hline\hline
\vocab{homolytic fission} & \vocab{heterolytic fission} \\
\hline
Breaking of covalent bond such that shared pair of electrons are split equally between the two atoms, which forms free radicals & 
Breaking of covalent bond such that shared pair of electrons are split unequally between the two atoms after bond broken, which forms ions \\
\hline
Movement of single electron (half arrow): &

Movement of electron pair (full arrow): \\
\hline\hline
\end{tabular}
\end{table}

Types of reactions\footnote{Do not use these terms when asked for the type of reaction! Give answers such as free radical substitution, electrophilic addition etc.}
\begin{table}[H]
\centering
\begin{tabular}{p{3cm}p{8cm}p{4cm}}
\hline\hline
\textbf{Reaction} & \textbf{Description} & \textbf{Bonds} \\
\hline
\textbf{addition} & Two reactants added together to form one single product & One $\pi$ bond broken, two $\sigma$ bonds formed \\
\textbf{substitution} & An atom / a group of atoms replaces another atom / group of atoms & One $\sigma$ bond broken, one $\sigma$ bond formed \\
\textbf{elimination} & Two atoms / groups of atoms from adjacent atoms removed from one molecule & Two $\sigma$ bonds broken, one $\pi$ bond formed \\
\textbf{condensation} & Two molecules react to form a larger molecule + elimination of simple molecule &  \\
\textbf{hydrolysis} & Reaction with water / \ce{H+} / \ce{OH-} &  \\
\textbf{oxidation} & Increase in oxidation state, oxidised by [O] &  \\
\textbf{reduction} & Decrease in oxidation state, reduced by [H] &  \\
\hline\hline
\end{tabular}
\end{table}
\pagebreak

\subsection{Hybridisation}
One of 2s electron is promoted to empty 2p orbital (from ground state to excited state)
→ 4 singly occupied orbitals for forming 4 covalent bonds

\begin{exercise}{A-Level 2021/III/2(e)}{}
Describe and explain, in terms of orbital overlap, the shape and bonding in an ethyne molecule \ce{C2H2}. \hfill \textbf{[3]}
\end{exercise}
\begin{proof}[Answer]
Ethyne molecule is linear whereby all four atoms lie in a straight line. 

Both carbons are sp-hybridised. In an sp-hybridised carbon, 2s orbital combines with 2p$_x$ orbital to form two hybrid orbitals that are oriented at an angle of $180\degree$ with respect to each other, along $x$-axis. 2p$_y$ and 2p$_z$ orbitals remain unhybridised, orientated perpendicularly along $y$- and $z$-axes respectively.

\ce{C-C} $\sigma$ bond is formed by the overlap of one sp orbital from each carbon atom. Two \ce{C-H} $\sigma$ bonds are formed by overlap of the second sp orbital on each carbon with 1s orbital on hydrogen atom. \ce{C#C} bond is formed by overlap of 2p$_y$ and 2p$_z$ orbitals of each carbon atom overlap sideways to form two $\pi$ bonds between the carbons.
\end{proof}

\begin{exercise}{A-Level 2017/II/7(a)(iii)}{}
Describe the hybridisation of the orbitals in, and the bonds between, the carbon atoms within a naphthalene molecule. \hfill \textbf{[3]}
\end{exercise}
\begin{proof}[Answer]
One of the 2s electrons in each carbon atom is promoted to the vacant 2p orbital, and the 2s orbital and two 2p orbitals hybridise to give three sp$^2$ hybrid orbitals.

Adjacent carbon atoms are bonded by a $\sigma$ bond through head-on overlapping of sp$^2$ hybrid orbitals, and a $\pi$ bond through sideways overlapping of unused p$_z$ orbital of each carbon atom, that is \ce{C=C} bond. Due to delocalisation of electrons, all carbon atoms are covalently bonded by sp$^2$--sp$^2$ overlap.
\end{proof}
\pagebreak

\section{Isomerism}
\begin{defn}{Isomers}{}
Compounds having same molecular formula but different arrangement of atoms.
\end{defn}

\begin{enumerate}
\item \textbf{Structural isomerism}

Same molecular formula but different arrangements of atoms

\begin{itemize}
\item \textbf{Chain isomerism}: different arrangement of carbon chain (straight or branched)
\begin{center}
\chemname{\chemfig{-[:30]-[:-30]-[:30]}}{butane}
\qquad
\chemname{\chemfig{-[:30](-[:90])-[:-30]}}{methylpropane}
\end{center}

\item \textbf{Positional isomerism}: different position of functional group
\begin{center}
\chemname{\chemfig{-[:30]-[:-30]-[:30]OH}}{propan-1-ol}
\qquad
\chemname{\chemfig{-[:30](-[:90]OH)-[:-30]}}{propan-2-ol}
\end{center}

\item \textbf{Functional group isomerism}: different functional group
\begin{center}
\chemname{\chemfig{=[:30]-[:-30]-[:30]}}{but-1-ene}
\qquad
\chemname{\chemfig{*4(----)}}{cyclobutane}
\end{center}
\end{itemize}

\item \textbf{Stereoisomerism}

\begin{itemize}
\item \textbf{Cis-trans isomerism}
\begin{center}
\chemname{\chemfig{(-[:120])=-[:60]}}{cis-but-2-ene}
\qquad
\chemname{\chemfig{(-[:240])=-[:60]}}{trans-but-2-ene}
\end{center}

\item \textbf{Enantiomerism}
\begin{center}
\chemfig{(<:[:210]B)(<[:255]C)(-[:330]D)-[:90]A}
\qquad
\chemfig{(-[:210]D)(<[:285]C)(<:[:330]B)-[:90]A}
\end{center}
\end{itemize}
\end{enumerate}



\begin{exercise}{A-Level 2021/III/4(e)(ii)}{}
The Diels-Alder reaction is a one-step reaction between a diene and a substituted alkene to form a substituted cyclohexene X.

\vspace{.5cm}
\schemestart
\chemfig{(=[:30])-[:270]=[:-30]}
\+
\chemfig{([:270]=)-[:30]CO_2R}
\arrow(.mid east--.mid west){->[heat]}
\chemfig{*6(---(-CO_2R)--=)}
\schemestop
\vspace{.5cm}

Compound X has no effect on the plane of polarised light. Explain your reasoning. \hfill \textbf{[1]}
\end{exercise}

\begin{proof}[Answer]
Diene can attack sp$^2$ carbon on \ce{C=C} bond of the substituted alkene from both top and bottom of the plane with equal probability, since it is trigonal planar.

As such, a racemic mixture is obtained which is optically inactive since optical activity of one enantiomer exactly cancels the other.
\end{proof}
\pagebreak

\part{Hydrocarbons}
\section{Alkanes}
\subsection*{Chemical properties}
Generally unreactive
\begin{itemize}
\item Fully saturated: all carbon atoms are sp$^3$ hybridised and \ce{C-C}, \ce{C-H} bonds are strong and difficult to break, so alkanes \emph{do not undergo addition}.
\item \ce{C-H} bonds are non-polar: lack electron-rich ($\delta-$) and electron-deficient ($\delta+$) sites, so alkanes are \emph{unreactive towards polar reagents}.
\end{itemize}

Small cycloalkanes (e.g. cyclopropane, cyclobutane) are unstable due to \textbf{ring strain}, as molecules are forced into smaller bond angles.

\subsection{Mechanism: Free radical substitution}
\begin{mechanism}{Free radical substitution}{}
\begin{enumerate}[leftmargin=0.65in,label=\textbf{Step \arabic*:}]
\item Initiation
\item Propagation
\item Termination
\end{enumerate}

Overall equation:

\vspace{.5cm}
\schemestart
\chemfig{-[:30]-[:-30]}
\arrow(.mid east--.mid west){->[\ce{X2}, UV]}[,2.1]
\chemfig{(-[:90]X)(-[:210])-[:-30]}
\schemestop
\end{mechanism}

\subsection{Reactions}
\begin{enumerate}
\item \textbf{Free radical substitution}
\end{enumerate}

\subsection{Preparation}
\begin{enumerate}
\item Reduction of alkene
\end{enumerate}
\pagebreak

\section{Alkenes}
\subsection*{Chemical properties}
Generally reactive
\begin{itemize}
\item Electron rich C=C bond is easily accessible to approaching reactants ($\pi$ electrons are located above and below the plane of bond) → act as nucleophiles, attract electrophiles OR induce dipoles in approaching molecules to form electrophiles

\item Carbon atoms in C=C bond are unsaturated: sp2 hybridised, can bond with one more atom → undergo electrophilic addition
\end{itemize}

\subsection{Mechanism: Electrophilic addition}
\begin{mechanism}{Electrophilic addition}{}
\begin{enumerate}[leftmargin=0.65in,label=\textbf{Step \arabic*:}]
\item Addition of electrophile

formation of carbocation (Markovnikov’s rule)

\item Ions combine
\end{enumerate}
\end{mechanism}

\subsection{Reactions}
\begin{enumerate}
\item \textbf{Electrophilic addition}
	\begin{itemize}
	\item Addition of halogen
	\item Addition of halogen in water\footnote{If water is used as solvent, \ce{H2O} can act as nucleophile to attack product formed.}
	\item Addition of hydrogen halide
	\item Addition of water
	\end{itemize}
\item \textbf{Reduction}
\item \textbf{Oxidation}
	\begin{itemize}
	\item Mild oxidation
	\item Strong oxidation (to form carbon dioxide, carboxylic acid, or ketone)
	\end{itemize}
\end{enumerate}

\begin{remark}
Remember how to write oxidation / reduction equations.
\end{remark}

\subsection{Preparation}
\begin{enumerate}
\item Elimination of alcohol
\item Elimination of halogenoalkane
\end{enumerate}
\pagebreak

\section{Benzene}
\subsection*{Chemical properties}
Resonance stability
\begin{itemize}
\item Due to overlapping p-orbitals, delocalised $\pi$ electron cloud above and below plane of ring. Benzene is resonance stabilised, resonance structure of benzene is very stable i.e. aromaticity.
\item Do not undergo reactions that destroy resonance stability (even through electron-rich due to presence of $\pi$ electrons), e.g. electrophilic addition. Hence benzene only undergoes electrophilic substitution, remain resonance stabilised.
\end{itemize}

\subsection{Mechanism: Electrophilic substitution}
\begin{mechanism}{Electrophilic substitution}{}

\begin{enumerate}[leftmargin=0.65in,label=\textbf{Step \arabic*:}]
\item Generation of strong electrophile
\item Addition into benzene ring

position dependent on group which is already present

\item Deprotonation
\end{enumerate}
\end{mechanism}

\subsection{Reactions}
\begin{enumerate}
\item \textbf{Electrophilic substitution}
	\begin{itemize}
	\item Electrophilic substitution of halogen (halogenation)
	\item Electrophilic substitution of nitro group (nitration)
	\item Electrophilic substitution of alkyl group (Friedel-Crafts alkylation)
	\end{itemize}
\item \textbf{Reduction}
\end{enumerate}
\pagebreak

\section{Methylbenzene}
\subsection*{Chemical properties}

\subsection{Mechanism: Electrophilic substitution}
Same as above.

\subsection{Reactions}
\begin{enumerate}
\item \textbf{Electrophilic substitution}
	\begin{itemize}
	\item Electrophilic substitution of halogen (halogenation)
	\item Electrophilic substitution of nitro group (nitration)
	\end{itemize}
\item \textbf{Free radical substitution} (side chain)
\item \textbf{Oxidation} (side chain)
\end{enumerate}

\subsection{Preparation}
\begin{enumerate}
\item Electrophilic substitution of benzene
\end{enumerate}
\pagebreak

\part{Halogen Derivatives}
\section{Halogenoalkane}
\subsection{Mechanism: Nucleophilic substitution}
\begin{mechanism}{Unimolecular nucleophilic substitution (S$_\text{N}$1)}{}
\begin{enumerate}[leftmargin=0.65in,label=\textbf{Step \arabic*:}]
\item Formation of carbocation intermediate

%\schemestart
%\chemfig{-(-[:30])[:-30]X}
%\arrow(.mid east--.mid west){->[slow]}
%\chemfig{-(-[:-30])[:-150]}
%\schemestop

\item Attack of carbocation by nucleophile
\end{enumerate}
\end{mechanism}

\begin{mechanism}{Bimolecular nucleophilic substitution (S$_\text{N}$2)}{}
\begin{enumerate}[leftmargin=0.65in,label=\textbf{Step \arabic*:}]
\item Nucleophile attacks from opposite side of halogen atom

Pentavalent transition state formed, where both nucleophile and halogen are partially bonded to carbon atom, both bond breaking and bond forming process take place simultaneously.
\end{enumerate}
\end{mechanism}

\begin{table}[H]
\centering
\begin{tabular}{p{0.5\textwidth}p{0.5\textwidth}}
\hline\hline
S$_\text{N}$1 & S$_\text{N}$2 \\
\hline
one molecule in first step & two molecules in first step \\
two steps & one step \\
\textbf{Electronic:} tertiary halogenoalkane gives stable tertiary carbocation intermediate & \textbf{Electronic:} methyl and primary halogenoalkane give less stable methyl carbocation and primary cartion intermediate \\
\textbf{Steric:} tertiary halogenoalkane has three bulky groups which hinder approach of nucleophile to electron-deficient carbon atom, more steric hindrance & \textbf{Steric:} methyl and primary halogenoalkane have no or only one alkyl group which allows easy approach of nucleophile to electron-deficient carbon atom, less steric hindrance \\
\textbf{Stereochemistry:} inversion of stereochemical configuration (for chiral reactants) & \textbf{Stereochemistry:} racemic mixture (for chiral reactants) \\
\hline\hline
\end{tabular}
\end{table}

Exceptions (due to other electronic and steric considerations)
\begin{itemize}
\item 
\end{itemize}

\subsection{Reactions}
\begin{enumerate}
\item \textbf{Nucleophilic substitution}
	\begin{itemize}
	\item Nucleophilic substitution of \ce{OH-}
	\item Nucleophilic substitution of \ce{CN-}
	\item Nucleophilic substitution of \ce{NH3} (step-up reaction)
	
	Acidic hydrolysis, basic hydrolysis, reduction
	\item Nucleophilic substitution of \ce{RO-}, formed from alcohol [FYI]
	\end{itemize}
\item \textbf{Elimination}
\end{enumerate}

\subsection{Preparation}
\begin{enumerate}
\item Free radical substitution of alkane
\item Electrophilic addition of alkene
\item Nucleophilic substitution of alcohol
\end{enumerate}

\subsection{Reactivities of Halogenoalkanes}
Down the group, atomic orbital of halogen atom becomes more diffused, effectiveness of orbital overlap decreases, strength of \ce{C-X} bond decreases down the group, relative ease of breaking \ce{C-X} bond increases, reactivity of \ce{R-X} towards nucleophilic substitution increases

\begin{remark}
Do not use bond polarity to explain.
\end{remark}

\subsection{Uses}
Fluoroalkanes and fluorohalogenoalkanes are generally stable and unreactive (chemically inert) due to strong \ce{C-F} bond (high bond energy), used as \textbf{inert materials} in fire extinguisher, refrigerant, aerosol propellant etc.

\textbf{Chlorofluorocarbons (CFCs)} lead to ozone depletion, by free radical substitution:
\[ \ce{CFCl3 -> Cl. + .CFCl2} \]
\[ \ce{Cl. + O3 -> ClO. + O2} \]
\[ \ce{ClO. + O. -> Cl. + O2} \]
\pagebreak

\section{Halogenoarene}
\subsection*{Chemical properties}
Unreactive towards nucleophilic substitution
\begin{itemize}
\item \textbf{Electronic:}

p orbital of halogen atom overlaps with p-orbitals of carbon atoms on benzene ring. Lone pair of electrons in p-orbital of halogen atom delocalises into benzene ring to form delocalised $\pi$ electron cloud.

\emph{Partial double bond character} in \ce{C-X} bond, more energy required to break stronger \ce{C-X} bond to displace halogen atom.

\item \textbf{Steric:}

Rear side of \ce{C-X} bond is blocked by bulky benzene ring. $\pi$ electron cloud of benzene ring repulses lone pair of electrons of nucleophile, difficult for nucleophile to attack.
\end{itemize}

\subsection{Reactions}

\subsection{Preparation}
\pagebreak

\part{Hydroxy Compounds}
\section{Alcohol}
\subsection{Acidity}

\subsection{Reactions}
\begin{enumerate}
\item \textbf{Redox reaction} (acid-metal displacement)
\item \textbf{Condensation reaction}
	\begin{itemize}
	\item Condensation using carboxylic acid
	\item Condensation using acyl chloride
	\end{itemize}
\item \textbf{Nucleophilic substitution}
\item \textbf{Elimination}
\item \textbf{Oxidation}
	\begin{itemize}
	\item Oxidation of $1\degree$ and $2\degree$ alcohols
	\item Oxidation of \ce{CH3CH(OH)-R} alcohols (tri-iodoform test)
	\end{itemize}
\end{enumerate}

\subsection{Preparation}
\begin{enumerate}
\item Electrophilic addition of alkene
\item Nucleophilic substitution of halogenoalkane
\item Reduction of carbonyl compound
\end{enumerate}
\pagebreak

\section{Phenol}
\subsection{Acidity}
(i) its acidity; reaction with bases and sodium
(ii) nitration of, and bromination of, the aromatic ring

\subsection{Reactions}
\begin{enumerate}
\item Electrophilic substitution
	\begin{itemize}
	\item Electrophilic substitution of halogen
	\item Electrophilic substitution of nitro group (nitration)
	\end{itemize}
\item Redox reaction (acid-metal displacement)
\item Neutralisation
\item Condensation
\item Complex formation
\end{enumerate}

explain the relative acidities of water, phenol and ethanol in aqueous medium (interpret as Brønsted-Lowry acids)
\pagebreak

\part{Carbonyl Compounds}
\section{Aldehydes}
(i) oxidation to carboxylic acid
(ii) nucleophilic addition with hydrogen cyanide
(iii) characteristic tests for aldehydes
\pagebreak

\section{Ketones}
(i) nucleophilic addition with hydrogen cyanide
(ii) characteristic tests for ketones

(a) describe the formation of aldehydes and ketones from, and their reduction to, primary and secondary
alcohols respectively
(b) describe the mechanism of the nucleophilic addition reactions of hydrogen cyanide with aldehydes and
ketones
(c) explain the differences in reactivity between carbonyl compounds and alkenes towards nucleophilic
reagents, such as lithium aluminium hydride and hydrogen cyanide
(d) describe the use of 2,4-dinitrophenylhydrazine (2,4-DNPH) to detect the presence of carbonyl compounds
(e) deduce the nature (aldehyde or ketone) of an unknown carbonyl compound from the results of simple tests
(i.e. Fehling’s and Tollens’ reagents; ease of oxidation)
(f) deduce the presence of a CH3CO– group in a carbonyl compound from its reaction with alkaline aqueous iodine to form tri-iodomethane
\pagebreak

\part{Carboxylic Acid and Derivative}
\section{Carboxylic acids}
(i) formation from primary alcohols and nitriles
(ii) salt, ester and acyl chloride formation
\pagebreak

\section{Acyl chlorides}
(i) ease of hydrolysis compared with alkyl and aryl chlorides
(ii) reaction with alcohols, phenols and primary amines
\pagebreak

\section{Esters}
(i) formation from carboxylic acids and from acyl chlorides
(ii) hydrolysis (under acidic and under basic conditions)


(a) describe the formation of carboxylic acids from alcohols, aldehydes and nitriles
(b) describe the reactions of carboxylic acids in the formation of:
(i) salts
(ii) esters on condensation with alcohols, using ethyl ethanoate as an example
(iii) acyl chlorides, using ethanoyl chloride as an example
(iv) primary alcohols, via reduction with lithium aluminium hydride, using ethanol as an example
(c) explain the acidity of carboxylic acids and of chlorine-substituted ethanoic acids in terms of their structures
(d) describe the hydrolysis of acyl chlorides
(e) describe the condensation reactions of acyl chlorides with alcohols, phenols and primary amines
(f) explain the relative ease of hydrolysis of acyl chlorides, alkyl chlorides and aryl chlorides
(g) describe the formation of esters from the condensation reaction of acyl chlorides, using phenyl benzoate as an example
(h) describe the acid and base hydrolysis of esters
\pagebreak

\part{Nitrogen Compounds}
\section{Amines}
(i) their formation
(ii) salt formation
(iii) other reactions of phenylamine
\pagebreak

\section{Amides}
(i) formation from acyl chlorides
(ii) neutrality of amides
(iii) hydrolysis (under acidic and under basic conditions)
\pagebreak

\section{Amino acids}
(i) their acid and base properties
(ii) zwitterion formation
\pagebreak

\section{Proteins}
(i) formation of proteins
(ii) hydrolysis of proteins

(a) describe the formation of amines as exemplified by ethylamine (through amide and nitrile reduction; see also Section 11.4) and by phenylamine (through the reduction of nitrobenzene)
(b) describe the reaction of amines in the formation of salts
(c) describe and explain the basicity of primary, secondary and tertiary amines in the gaseous phase (interpret
as Lewis bases)
(d) explain the relative basicities of ammonia, ethylamine and phenylamine in aqueous medium, in terms of
their structures
(e) describe the reaction of phenylamine with aqueous bromine
(f) describe the formation of amides from the condensation reaction between RNH2 and R'COCl
(g) explain why an amide is neutral in terms of delocalisation of the lone pair of electrons on nitrogen
(h) describe the chemistry of amides, exemplified by the following reactions:
(i) hydrolysis on treatment with aqueous alkali or acid
(ii) reduction to amines with lithium aluminium hydride
(i) describe the acid/base properties of amino acids and the formation of zwitterions
[knowledge of isoelectric points is not required]
(j) describe the formation of peptide (amide) bonds between $\alpha$-amino acids, and hence explain protein formation
(k) describe the hydrolysis of proteins
\end{document}
