\section{Mole Concept and Stoichiometry}
\subsection{Relative masses}
\begin{defn}{Relative isotopic mass}{}
Mass of one atom of isotope of element relative to $\frac{1}{12}$ of mass of \ce{C-12} atom.
\end{defn}

\begin{defn}{Relative atomic mass $A_r$}{}
Average mass of atoms of element in isotopic mixture relative to $\frac{1}{12}$ of mass of \ce{C-12} atom.
\end{defn}

Given $i$ isotopes, $A_r$ is the weighted average of all isotopes (so it is usually not an integer).
\begin{equation}
A_r = \frac{\sum_i m_i \times \text{Abundance}_i}{\sum_i \text{Abundance}_i}
\end{equation}

\begin{defn}{Relative molecular mass $M_r$}{}
Average mass of one molecule relative to $\frac{1}{12}$ of mass of \ce{C-12} atom.
\end{defn}

$M_r$ is the sum of the $A_r$ of the atoms shown in the molecular formula.
\begin{equation}
M_r = \sum_i (A_r)_i
\end{equation}

\begin{defn}{Relative formula mass $M_r$}{}
Average mass of one formula unit of substance relative to $\frac{1}{12}$ of mass of \ce{C-12} atom.
\end{defn}

\begin{remark}
A formula unit is the smallest collection of atoms from which the formula of an ionic compound can be established. It is equal to the sum of the $A_r$ of the atoms shown in the formula unit.
\end{remark}

\begin{exercise}{}{}
Determine the $A_r$ of chlorine given that there exist two isotopes, \ce{^{35}Cl} and \ce{^{37}Cl}, with percentage isotopic abundance 75\% and 25\%, respectively.
\end{exercise}

\begin{solution}
To calculate $A_r$, we need to consider the relative amount of each isotope. To calculate a weighted average,
\[ A_r\text{ of chlorine} = \frac{75(35)+25(37)}{75+25} = \boxed{35.5} \]
\end{solution}

\subsection{Mole}
\begin{defn}{Mole}{}
One mole contains exactly $6.02 \times 10^{23}$ (or Avogadro constant) elementary entities.
\end{defn}

Avogadro constant $L=6.02 \times 10^{23}\,\unit{mol^{-1}}$ 

\begin{defn}{Molar mass}{}
Mass of one mole of substance.
\end{defn}

\begin{defn}{Avogadro’s Law}{}
Equal volumes of all gases, under same conditions of temperature and pressure, contain same number of molecules/ atoms.
\end{defn}

\begin{defn}{Molar volume}{}
Volume occupied by one mole of gas.
\end{defn}

Molar volume at r.t.p. = $24.0\,\unit{dm^3\,mol^{-1}}$

Molar volume at s.t.p. = $22.7\,\unit{dm^3\,mol^{-1}}$

\begin{defn}{Standard solution}{}
Solution which contains known amount of solute in given volume of solution (i.e. one with a known concentration).
\end{defn}

Dilution:
\begin{equation}
C_0V_0 = C_dV_d
\end{equation}

\subsection{Oxidation number}
Rules to assign oxidation number
\begin{enumerate}
\item ON of element = 0
\item ON of H in compound = +1 (except in metal hydrides)
\item ON of O in compound = -2 (except in peroxides, superoxides)
\item ON of more electronegative atom = -ve
\item ON of less electronegative atom = +ve
\item ON of uncharged compound = sum of individual ON = 0
\end{enumerate}

\subsection{Redox reactions}
To construct a redox equation in acidic medium,
\begin{enumerate}
\item Construct unbalanced oxidation and reduction half-equations
\item Balance all elements except \ce{H} and \ce{O}
\item Balance \ce{O} atoms by adding \ce{H2O}
\item Balance \ce{H} atoms by adding \ce{H+} ions
\item Balance charges by adding electrons
\item Add both half-equations
\end{enumerate}

To construct a redox equation in alkaline medium,
\begin{enumerate}[resume]
\item Neutralise \ce{H+} ions by adding \ce{OH-} ions, combine \ce{H+} and \ce{OH-} to form \ce{H2O}
\end{enumerate}

\subsubsection{Half equations}
\textbf{Manganate(VII)} as oxidising agent
\[ \ce{MnO4^- + 8 H^+ + 5 e^- -> Mn^2+ + 4 H2O} \]
\textbf{Dichromate(VI)} as oxidising agent
\[ \ce{Cr2O7^2- + 14 H^+ + 6 e^- -> 2 Cr^3+ + 7 H2O} \]
\textbf{Thiosulfate} as reducing agent
\[ \ce{I2 + 2 S2O3^2- -> 2 I^- + S4O6^2-} \]

Calculate empirical and molecular formulae

Back-titration

\subsection{Combustion}
Complete combustion of hydrocarbon:
\begin{center}
\ce{C_xH_y + $\brac{x+\dfrac{y}{4}}$ O2(g) -> $x$ CO2(g) + $\dfrac{y}{2}$ H2O(l)}
\end{center}
\pagebreak