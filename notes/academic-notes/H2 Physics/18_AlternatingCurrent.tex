\section{Alternating Current}
\begin{defn}{Alternating current (a.c.)}{}
Current that varies periodically with time in magnitude and direction. 

(Polarity of the voltage source constantly changes)
\end{defn}

\subsection{Characteristics}
\textbf{Period} $T$: time taken for the current to undergo one complete cycle.

\textbf{Frequency} $f$: number of complete cycles undergone by the current per unit time.

\textbf{Angular frequency} $\omega$: frequency in terms of radians per unit time rather than cycles per unit time.
\[ \omega = \frac{2\pi}{T} \]

\textbf{Peak value} $I_0$: maximum magnitude of the current attained in each cycle.

\textbf{Peak-to-peak value} $I_{pp}$: difference between the maximum and minimum values of the current within one cycle.

For a sinusoidal wave, 
\[ I_{pp} = 2I_0 \]

Mean Value $\langle I\rangle$: average value of a.c. over a given time interval.

Root mean square value $I_\text{r.m.s.}$: value of alternating current that is equal to the steady direct current which would dissipate heat at the same average rate in a given resistor.

The most commonly encountered form of a.c. is the \textbf{sinusoidal} form, that is, it varies with time according to a sine or cosine function.

\begin{equation}
I = I_0 \sin \omega t
\end{equation}

Similarly, potential difference across the resistor is given by 
\[ V = V_0 \sin \omega t \]



\subsection{Transformer}
\subsubsection{Functioning}

\subsubsection{Turns ratio}

\subsubsection{Power loss}
Practical transformers lose power through
\begin{itemize}
\item \textbf{Joule heating}

The wires used for the windings of the coils have resistance and so heating occurs, resulting in power loss $P=I^2R$. Thicker wires made of material with low resistivity (i.e. high purity copper) are used to reduce this power loss.

\item \textbf{Eddy currents}

The alternating magnetic flux induces eddy currents in the iron core and cause heating. This effect is reduced by laminating the iron core. Laminations reduce the area of circuits in the core, and thus reduce the e.m.f. induced and current flowing within the core, which leads to a reduction in the energy lost.

\item \textbf{Hysteresis loss}

Magnetisation of the core is repeatedly reversed by the alternating magnetic field. The energy required to magnetise the core (while the current is increasing) is not entirely recovered during demagnetisation. The difference in energy is lost as heat in the core. The energy loss is kept to a minimum by using a magnetic material with low hysteresis loss.

\item \textbf{Flux leakage}

The flux due to the primary may not all link to the secondary coil if the coil is badly designed or has air gaps in it. When flux is “leaked “to the surrounding, power is loss and thus not all the power from the primary coil can be transferred to the secondary coil.
\end{itemize}


\subsection{Rectification with a diode}

\pagebreak

\subsection*{Problems}

\pagebreak